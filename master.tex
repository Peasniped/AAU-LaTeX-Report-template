%
% Dette dokument bruges til at "samle" rapporten ud fra de individuelle sektioner i .\sections\
%

\documentclass[11pt,twoside,a4paper,openright]{report}

%%
%% Preamble er en slags side med indstillinger for hele rapporten
%%

%%%%%%%%%%%%%%%%%%%%%%%%%%%%%%%%%%%%%%%%%%%%%%%%
% Language, Encoding and Fonts
% http://en.wikibooks.org/wiki/LaTeX/Internationalization
%%%%%%%%%%%%%%%%%%%%%%%%%%%%%%%%%%%%%%%%%%%%%%%%
% Select encoding of your inputs. Depends on
% your operating system and its default input
% encoding. Typically, you should use
%   Linux  : utf8 (most modern Linux distributions)
%            latin1 
%   Windows: ansinew
%            latin1 (works in most cases)
%   Mac    : applemac
% Notice that you can manually change the input
% encoding of your files by selecting "save as"
% an select the desired input encoding. 
\usepackage[utf8]{inputenc}
\usepackage{ifthen}
% Make latex understand and use the typographic
% rules of the language used in the document.
\usepackage[danish,english]{babel}
% Use the palatino font
\usepackage[sc]{mathpazo}
\linespread{1.05}         % Palatino needs more leading (space between lines)
% Choose the font encoding
\usepackage[T1]{fontenc}

%%%%%%%%%%%%%%%%%%%%%%%%%%%%%%%%%%%%%%%%%%%%%%%%
% Graphics and Tables
% http://en.wikibooks.org/wiki/LaTeX/Importing_Graphics
% http://en.wikibooks.org/wiki/LaTeX/Tables
% http://en.wikibooks.org/wiki/LaTeX/Colors
%%%%%%%%%%%%%%%%%%%%%%%%%%%%%%%%%%%%%%%%%%%%%%%%

% load a colour package
\usepackage[table]{xcolor}
\definecolor{aaublue}{RGB}{33,26,82}% dark blue
\definecolor{aaublue}{RGB}{33,26,82}
\definecolor{white}{RGB}{255,255,255}
\definecolor{black}{RGB}{0,0,0}

% Indlæs pakke der laver bedre tabeller
\usepackage{float}
\usepackage{tabularray}
% How to use Tabularray:
% https://www.latex-tables.com/ressources/tabularray.html
\usepackage{tabularx}

%%% Indlæs pakke der gør det muligt at lave figurer/tabeller ved siden af teksten
% Ligesom i Word hvor man kan formatere billedet som 'Tæt'
\usepackage{wrapfig}
%% Brugseksempel:
% Skriv noget tekst her som skal gå rundt om billedet
% \begin{wrapfigure}{r}{7.5cm} % <-- wraptable for tabel
%   ## Billede her ##
% \end{wrapfigure} 

% The standard graphics inclusion package
\usepackage{graphicx}

% Set up how figure and table captions are displayed
\usepackage{caption}
\captionsetup{%
  font=footnotesize,% set font size to footnotesize
  labelfont=bf % bold label (e.g., Figure 3.2) font
}
% Make the standard latex tables look so much better
\usepackage{array,booktabs}
% Enable the use of frames around, e.g., theorems
% The framed package is used in the example environment
\usepackage{framed}

% Adds support for full page background picture
\usepackage{watermark}
%\usepackage[contents={},color=gray]{background} % <--- Denne linje laver everypage-warning

%%%%%%%%%%%%%%%%%%%%%%%%%%%%%%%%%%%%%%%%%%%%%%%%
% Mathematics
% http://en.wikibooks.org/wiki/LaTeX/Mathematics
%%%%%%%%%%%%%%%%%%%%%%%%%%%%%%%%%%%%%%%%%%%%%%%%
% Defines new environments such as equation,
% align and split 
\usepackage{amsmath}
% Adds new math symbols
\usepackage{amssymb}
% Use theorems in your document
% The ntheorem package is also used for the example environment
% When using thmmarks, amsmath must be an option as well. Otherwise \eqref doesn't work anymore.
\usepackage[framed,amsmath,thmmarks]{ntheorem}

%%%%%%%%%%%%%%%%%%%%%%%+%%%%%%%%%%%%%%%%%%%%%%%%%
% Page Layout
% http://en.wikibooks.org/wiki/LaTeX/Page_Layout
%%%%%%%%%%%%%%%%%%%%%%%%%%%%%%%%%%%%%%%%%%%%%%%%
% Change margins, papersize, etc of the document

% Marginer optimeret til bog
% \usepackage[inner=40mm, outer=25mm, top=20mm, bottom=30mm,]{geometry}

% Marginer optimeret til web
\usepackage[inner=20mm, outer=20mm, top=20mm, bottom=35mm,]{geometry}
  
% Modify how \chapter, \section, etc. look
% The titlesec package is very configureable
\usepackage{titlesec}

% Old chapter title format:
%\titleformat{\chapter}[display]{\normalfont\huge\bfseries}{\chaptertitlename: \thechapter}{20pt}{\Huge}
% New chapter title format:
\titleformat{\chapter}[hang]{\normalfont\huge\bfseries}{\chaptertitlename\ \thechapter:}{20pt}{\Huge}

\titleformat*{\section}{\normalfont\Large\bfseries}
\titleformat*{\subsection}{\normalfont\large\bfseries}
\titleformat*{\subsubsection}{\normalfont\normalsize\bfseries}
%\titleformat*{\paragraph}{\normalfont\normalsize\bfseries}
%\titleformat*{\subparagraph}{\normalfont\normalsize\bfseries}

% Change spacing before and after subsubsection titles
% \titlespacing*{<command>}{<left>}{<before-sep>}{<after-sep>}
\titlespacing*{\subsubsection}{0pt}{8pt}{-5pt}

% Clear empty pages between chapters
\let\origdoublepage\cleardoublepage
\newcommand{\clearemptydoublepage}{%
  \clearpage
  {\pagestyle{empty}\origdoublepage}%
}
\let\cleardoublepage\clearemptydoublepage

% Change the headers and footers
\usepackage{fancyhdr}
\pagestyle{fancy}
\fancyhf{} %delete everything
\renewcommand{\headrulewidth}{0pt} %remove the horizontal line in the header
\fancyhead[RE]{\small\nouppercase\leftmark} %even page - chapter title
\fancyhead[LO]{\small\nouppercase\rightmark} %uneven page - section title
\fancyhead[LE,RO]{\thepage} %page number on all pages
% Do not stretch the content of a page. Instead,
% insert white space at the bottom of the page
\raggedbottom
% Enable arithmetics with length. Useful when
% typesetting the layout.
\usepackage{calc}

% Package to use landscape-environment
\usepackage{pdflscape}

%%%%%%%%%%%%%%%%%%%%%%%%%%%%%%%%%%%%%%%%%%%%%%%%
% Bibliography
% http://en.wikibooks.org/wiki/LaTeX/Bibliography_Management
%%%%%%%%%%%%%%%%%%%%%%%%%%%%%%%%%%%%%%%%%%%%%%%%

\usepackage[backend=biber,
            style=numeric,
            sorting=none,
            bibencoding=utf8,
            urldate=iso,  % iso: 'YYYY-MM-DD', short|terse: 'MM/DD/YYYY', long|comp: 'MMM DD, YYYY'
            seconds=true
            ]{biblatex}
\addbibresource{sections/7- bibliography.bib}
\usepackage{csquotes}

% Add bibliography and index to the table of
% contents
\usepackage[nottoc]{tocbibind}
\usepackage{comment}

%%%%%%%%%%%%%%%%%%%%%%%%%%%%%%%%%%%%%%%%%%%%%%%%
% Misc
%%%%%%%%%%%%%%%%%%%%%%%%%%%%%%%%%%%%%%%%%%%%%%%%

% Used in New Titlepages
\usepackage{ragged2e}

%%%% Acronyms og Glossary
% Aktiverer funktionalitet til at skrive og forklare forkortelser/begreber
% Begreber/forkortelser defineres i 5- glossary.tex, og bruges med \ac{} igennem teksten

% Version 2 af acro, da den er omkring 10 gange hurtigere.
\usepackage{acro}[=v2]

% Setup for acro v2
\acsetup{only-used=true, page-style=none}

\ProvideAcroEnding {possessive} {'s} {'s}
    
\ExplSyntaxOn
\NewAcroCommand \acg
  {
    \acro_possessive:
    \acro_use:n {#1}
  }
\NewAcroCommand \acsg
  {
    \acro_possessive:
    \acro_short:n {#1}
  }
\NewAcroCommand \aclg
  {
    \acro_possessive:
    \acro_long:n {#1}
  }
\ExplSyntaxOff

% Fix acro error
\ExplSyntaxOn
\prop_new:N \l__acro_foreign_format_prop
\ExplSyntaxOff

\usepackage[toc]{glossaries}
\usepackage{longtable}

%% Add todo notes in the margin of the document
\usepackage[
%  disable, %turn off todonotes
  colorinlistoftodos, %enable a coloured square in the list of todos
  textwidth=\marginparwidth, %set the width of the todonotes
  ]{todonotes}
  
%% Allows to enumerate with roman numbers
\usepackage{enumitem}
% Eksempel:
% \begin{enumerate}[label=\textbf{\arabic*: }]
    % \item item 1
    % \item item 2
    % \item item 3
% \end{enumerate}

%% Lipsum Pakken
% Pakken tilføjer kommando til at lave dummy-teksten 'lorem ipsum dolor sit amet'
\usepackage{lipsum}
% Eksempel:
% Man skriver \lipsum[x] og på x's plads skriver man de afsnit som man vil have. Pakken indeholder 150 sætninger, så tallet skal være imellem de to [1-150].
% \lipsum[1]        % Her får vi det første afsnit af dummy-teksten
% \lipsum[]         % Hvis man ikke angiver et tal får man afsnit 1 til 7 (ca 1½ sides tekst)
% \lipsum[24-26]    % her får man afsnitene 24 til 26 

%% PDFpages-pakken
% Pakken tilføjer kommando til at indsætte PDF
\usepackage{pdfpages}
% Eksempel:
% Inkluder hele PDF'en
% \includepdf[pages=-]{myfile.pdf}
%
% Inkluder den første side i PDF'en
% \includepdf[pages={1}]{myfile.pdf}

%%%%%%%%%%%%%%%%%%%%%%%%%%%%%%%%%%%%%%%%%%%%%%%%
% Codeblocks
% Props til @ramlov
%%%%%%%%%%%%%%%%%%%%%%%%%%%%%%%%%%%%%%%%%%%%%%%%
\usepackage{listings}
\usepackage{xcolor}

%New colors defined below
\definecolor{codegreen}{rgb}{0,0.8,0}
\definecolor{codegray}{rgb}{0.65,0.65,0.65}
\definecolor{codepurple}{rgb}{0.58,0,0.82}
\definecolor{codeblue}{rgb}{0.5, 0.5, 0.9}
\definecolor{backcolour}{rgb}{0.90, 0.90, 0.9}

%Code listing style named "mystyle"
\lstdefinestyle{mystyle}{
  backgroundcolor=\color{backcolour},   
  commentstyle=\color{codegreen},
  keywordstyle=\color{magenta},
  numberstyle=\color{codegray},
  stringstyle=\color{codepurple},
  basicstyle=\ttfamily\footnotesize,
  breakatwhitespace=false,         
  breaklines=true,                 
  captionpos=b,                    
  keepspaces=true,                 
  numbers=left,                    
  numbersep=5pt,                  
  showspaces=false,                
  showstringspaces=false,
  showtabs=false,                  
  tabsize=2
}
\lstset{style=mystyle}

%%% Eksempel:
% \begin{lstlisting}[language=Python, caption=this is a caption]
    % code line 1
    % code line 2
% \end{lstlisting}
 % package inclusion and set up of the document
\input{setup/hyphenations.tex} % 
%%
%% Denne fil er til at lave macroer som kan kaldes i andre filer, hvis denne fil er defineret
%% Filen defineres med: %%
%% Denne fil er til at lave macroer som kan kaldes i andre filer, hvis denne fil er defineret
%% Filen defineres med: %%
%% Denne fil er til at lave macroer som kan kaldes i andre filer, hvis denne fil er defineret
%% Filen defineres med: \input{setup/macros.tex}
%%

%%%%%%%%%%%%%%%%%%%%%%%%%%%%%%%%%%%%%%%%%%%%%%%%
% Reference-macroer
%%%%%%%%%%%%%%%%%%%%%%%%%%%%%%%%%%%%%%%%%%%%%%%%

% myRef - Reference med angivelse af figur/tabel
\newcommand{\myRef}[2]{\textit{{#1} \ref{#2}}}

% myRefPage - Reference med angivelse af figur/tabel og sidetal
\newcommand{\myRefPage}[2]{\textit{{#1} \ref{#2} on page \pageref{#2}}}

%%%%%%%%%%%%%%%%%%%%%%%%%%%%%%%%%%%%%%%%%%%%%%%%
% Misc-macroer
%%%%%%%%%%%%%%%%%%%%%%%%%%%%%%%%%%%%%%%%%%%%%%%%

% myTodayKort - Print af dags dato på kort format (DD/MMM-YYYY)
\newcommand{\myTodayKort}{\ifnum\number\day<10 0\fi \number\day /\ifcase \month \or jan\or feb\or mar\or apr\or may \or jun\or jul\or aug\or sep\or oct\or nov\or dec\fi -\number \year} 

% myTodayLang - Print af dags dato på langt format (DD. MMMM YYYY)(engelske måneder)
\newcommand{\myTodayLang}{\ifnum\number\day<10 0\fi \number\day .\space\ifcase \month \or January\or February\or March\or April\or May \or June\or July\or August\or September\or October\or November\or December\fi \space\number \year} 

% myAppendixPageNumber - Nulstiller sidetal, så det specificeres for det enkelte bilag og formaterer sidetallet med prefix som sættes med parameteren.
\newcommand{\myAppendixPageNumbering}[1]{%
    \pagenumbering{arabic} % resets page-counter to 1
    \renewcommand*{\thepage}{{#1}\arabic{page}} 
}


\clearpage


%%%%%%%%%%%%%%%%%%%%%%%%%%%%%%%%%%%%%%%%%%%%%%%%
% Macros for the titlepage
%%%%%%%%%%%%%%%%%%%%%%%%%%%%%%%%%%%%%%%%%%%%%%%%
%Creates the aau titlepage
\newcommand{\aautitlepage}[3]{%
  {
    %set up various length
    \ifx\titlepageleftcolumnwidth\undefined
      \newlength{\titlepageleftcolumnwidth}
      \newlength{\titlepagerightcolumnwidth}
    \fi
    \setlength{\titlepageleftcolumnwidth}{0.48\textwidth-\tabcolsep}
    \setlength{\titlepagerightcolumnwidth}{\textwidth-2\tabcolsep-\titlepageleftcolumnwidth}
    %create title page
    \thispagestyle{empty}
    \noindent%
    \begin{tabular}{@{}ll@{}}
      \parbox{\titlepageleftcolumnwidth}{
        \iflanguage{danish}{%
          \includegraphics[width=\titlepageleftcolumnwidth-5mm]{media/AAUgraphics/aau_logo_da}
        }{%
          \includegraphics[width=\titlepageleftcolumnwidth-5mm]{media/AAUgraphics/aau_logo_en}
        }
      } &
      \parbox{\titlepagerightcolumnwidth}{\raggedleft\sf\small
        #2
      }\bigskip\\
       #1 &
      \parbox[t]{\titlepagerightcolumnwidth}{%
      \textbf{Abstract:}\bigskip\par
        \fbox{\parbox{\titlepagerightcolumnwidth-2\fboxsep-2\fboxrule}{%
          #3
        }}
      }\\
    \end{tabular}
    \vfill \bigskip
    \iflanguage{danish}{\noindent{\footnotesize\emph{Rapportens indhold er frit tilgængeligt, men offentliggørelse\\ (med kildeangivelse) må kun ske efter aftale med forfatterne.}}
    }{\noindent{\footnotesize\emph{The content of this report is freely available, but publication\\ (with reference) may only be pursued due to agreement with the author.}}
    }
    \clearpage
  }
}

%Create english project info
\newcommand{\englishprojectinfo}[8]{%
  \parbox[t]{\titlepageleftcolumnwidth}{
    \textbf{Title:}\\ #1\bigskip\par
    \textbf{Theme:}\\ #2\bigskip\par
    \textbf{Project Period:}\\ #3\bigskip\par
    \textbf{Project Group:}\\ #4\bigskip\par
    \textbf{Participant(s):}\\ #5\bigskip\par
    \textbf{Supervisor(s):}\\ #6\bigskip\par
    \textbf{Copies:} #7\bigskip\par
    \textbf{Number of pages:} \pageref{myLastPage}\bigskip\par
    \textbf{Date of Completion:}\\ #8
  }
}

%Create danish project info
\newcommand{\danishprojectinfo}[8]{%
  \parbox[t]{\titlepageleftcolumnwidth}{
    \textbf{Titel:}\\ #1\bigskip\par
    \textbf{Tema:}\\ #2\bigskip\par
    \textbf{Projektperiode:}\\ #3\bigskip\par
    \textbf{Projektgruppe:}\\ #4\bigskip\par
    \textbf{Deltager(e):}\\ #5\bigskip\par
    \textbf{Vejleder(e):}\\ #6\bigskip\par
    \textbf{Oplagstal:} #7\bigskip\par
    \textbf{Sidetal:} \pageref{myLastPage}\bigskip\par
    \textbf{Afleveringsdato:}\\ #8
  }
}

%%%%%%%%%%%%%%%%%%%%%%%%%%%%%%%%%%%%%%%%%%%%%%%%
% An example environment
%%%%%%%%%%%%%%%%%%%%%%%%%%%%%%%%%%%%%%%%%%%%%%%%
\theoremheaderfont{\normalfont\bfseries}
\theorembodyfont{\normalfont}
\theoremstyle{break}
\def\theoremframecommand{{\color{gray!50}\vrule width 5pt \hspace{5pt}}}
\newshadedtheorem{exa}{Example}[chapter]
\newenvironment{example}[1]{%
		\begin{exa}[#1]
}{%
		\end{exa}
}
%%

%%%%%%%%%%%%%%%%%%%%%%%%%%%%%%%%%%%%%%%%%%%%%%%%
% Reference-macroer
%%%%%%%%%%%%%%%%%%%%%%%%%%%%%%%%%%%%%%%%%%%%%%%%

% myRef - Reference med angivelse af figur/tabel
\newcommand{\myRef}[2]{\textit{{#1} \ref{#2}}}

% myRefPage - Reference med angivelse af figur/tabel og sidetal
\newcommand{\myRefPage}[2]{\textit{{#1} \ref{#2} on page \pageref{#2}}}

%%%%%%%%%%%%%%%%%%%%%%%%%%%%%%%%%%%%%%%%%%%%%%%%
% Misc-macroer
%%%%%%%%%%%%%%%%%%%%%%%%%%%%%%%%%%%%%%%%%%%%%%%%

% myTodayKort - Print af dags dato på kort format (DD/MMM-YYYY)
\newcommand{\myTodayKort}{\ifnum\number\day<10 0\fi \number\day /\ifcase \month \or jan\or feb\or mar\or apr\or may \or jun\or jul\or aug\or sep\or oct\or nov\or dec\fi -\number \year} 

% myTodayLang - Print af dags dato på langt format (DD. MMMM YYYY)(engelske måneder)
\newcommand{\myTodayLang}{\ifnum\number\day<10 0\fi \number\day .\space\ifcase \month \or January\or February\or March\or April\or May \or June\or July\or August\or September\or October\or November\or December\fi \space\number \year} 

% myAppendixPageNumber - Nulstiller sidetal, så det specificeres for det enkelte bilag og formaterer sidetallet med prefix som sættes med parameteren.
\newcommand{\myAppendixPageNumbering}[1]{%
    \pagenumbering{arabic} % resets page-counter to 1
    \renewcommand*{\thepage}{{#1}\arabic{page}} 
}


\clearpage


%%%%%%%%%%%%%%%%%%%%%%%%%%%%%%%%%%%%%%%%%%%%%%%%
% Macros for the titlepage
%%%%%%%%%%%%%%%%%%%%%%%%%%%%%%%%%%%%%%%%%%%%%%%%
%Creates the aau titlepage
\newcommand{\aautitlepage}[3]{%
  {
    %set up various length
    \ifx\titlepageleftcolumnwidth\undefined
      \newlength{\titlepageleftcolumnwidth}
      \newlength{\titlepagerightcolumnwidth}
    \fi
    \setlength{\titlepageleftcolumnwidth}{0.48\textwidth-\tabcolsep}
    \setlength{\titlepagerightcolumnwidth}{\textwidth-2\tabcolsep-\titlepageleftcolumnwidth}
    %create title page
    \thispagestyle{empty}
    \noindent%
    \begin{tabular}{@{}ll@{}}
      \parbox{\titlepageleftcolumnwidth}{
        \iflanguage{danish}{%
          \includegraphics[width=\titlepageleftcolumnwidth-5mm]{media/AAUgraphics/aau_logo_da}
        }{%
          \includegraphics[width=\titlepageleftcolumnwidth-5mm]{media/AAUgraphics/aau_logo_en}
        }
      } &
      \parbox{\titlepagerightcolumnwidth}{\raggedleft\sf\small
        #2
      }\bigskip\\
       #1 &
      \parbox[t]{\titlepagerightcolumnwidth}{%
      \textbf{Abstract:}\bigskip\par
        \fbox{\parbox{\titlepagerightcolumnwidth-2\fboxsep-2\fboxrule}{%
          #3
        }}
      }\\
    \end{tabular}
    \vfill \bigskip
    \iflanguage{danish}{\noindent{\footnotesize\emph{Rapportens indhold er frit tilgængeligt, men offentliggørelse\\ (med kildeangivelse) må kun ske efter aftale med forfatterne.}}
    }{\noindent{\footnotesize\emph{The content of this report is freely available, but publication\\ (with reference) may only be pursued due to agreement with the author.}}
    }
    \clearpage
  }
}

%Create english project info
\newcommand{\englishprojectinfo}[8]{%
  \parbox[t]{\titlepageleftcolumnwidth}{
    \textbf{Title:}\\ #1\bigskip\par
    \textbf{Theme:}\\ #2\bigskip\par
    \textbf{Project Period:}\\ #3\bigskip\par
    \textbf{Project Group:}\\ #4\bigskip\par
    \textbf{Participant(s):}\\ #5\bigskip\par
    \textbf{Supervisor(s):}\\ #6\bigskip\par
    \textbf{Copies:} #7\bigskip\par
    \textbf{Number of pages:} \pageref{myLastPage}\bigskip\par
    \textbf{Date of Completion:}\\ #8
  }
}

%Create danish project info
\newcommand{\danishprojectinfo}[8]{%
  \parbox[t]{\titlepageleftcolumnwidth}{
    \textbf{Titel:}\\ #1\bigskip\par
    \textbf{Tema:}\\ #2\bigskip\par
    \textbf{Projektperiode:}\\ #3\bigskip\par
    \textbf{Projektgruppe:}\\ #4\bigskip\par
    \textbf{Deltager(e):}\\ #5\bigskip\par
    \textbf{Vejleder(e):}\\ #6\bigskip\par
    \textbf{Oplagstal:} #7\bigskip\par
    \textbf{Sidetal:} \pageref{myLastPage}\bigskip\par
    \textbf{Afleveringsdato:}\\ #8
  }
}

%%%%%%%%%%%%%%%%%%%%%%%%%%%%%%%%%%%%%%%%%%%%%%%%
% An example environment
%%%%%%%%%%%%%%%%%%%%%%%%%%%%%%%%%%%%%%%%%%%%%%%%
\theoremheaderfont{\normalfont\bfseries}
\theorembodyfont{\normalfont}
\theoremstyle{break}
\def\theoremframecommand{{\color{gray!50}\vrule width 5pt \hspace{5pt}}}
\newshadedtheorem{exa}{Example}[chapter]
\newenvironment{example}[1]{%
		\begin{exa}[#1]
}{%
		\end{exa}
}
%%

%%%%%%%%%%%%%%%%%%%%%%%%%%%%%%%%%%%%%%%%%%%%%%%%
% Reference-macroer
%%%%%%%%%%%%%%%%%%%%%%%%%%%%%%%%%%%%%%%%%%%%%%%%

% myRef - Reference med angivelse af figur/tabel
\newcommand{\myRef}[2]{\textit{{#1} \ref{#2}}}

% myRefPage - Reference med angivelse af figur/tabel og sidetal
\newcommand{\myRefPage}[2]{\textit{{#1} \ref{#2} on page \pageref{#2}}}

%%%%%%%%%%%%%%%%%%%%%%%%%%%%%%%%%%%%%%%%%%%%%%%%
% Misc-macroer
%%%%%%%%%%%%%%%%%%%%%%%%%%%%%%%%%%%%%%%%%%%%%%%%

% myTodayKort - Print af dags dato på kort format (DD/MMM-YYYY)
\newcommand{\myTodayKort}{\ifnum\number\day<10 0\fi \number\day /\ifcase \month \or jan\or feb\or mar\or apr\or may \or jun\or jul\or aug\or sep\or oct\or nov\or dec\fi -\number \year} 

% myTodayLang - Print af dags dato på langt format (DD. MMMM YYYY)(engelske måneder)
\newcommand{\myTodayLang}{\ifnum\number\day<10 0\fi \number\day .\space\ifcase \month \or January\or February\or March\or April\or May \or June\or July\or August\or September\or October\or November\or December\fi \space\number \year} 

% myAppendixPageNumber - Nulstiller sidetal, så det specificeres for det enkelte bilag og formaterer sidetallet med prefix som sættes med parameteren.
\newcommand{\myAppendixPageNumbering}[1]{%
    \pagenumbering{arabic} % resets page-counter to 1
    \renewcommand*{\thepage}{{#1}\arabic{page}} 
}


\clearpage


%%%%%%%%%%%%%%%%%%%%%%%%%%%%%%%%%%%%%%%%%%%%%%%%
% Macros for the titlepage
%%%%%%%%%%%%%%%%%%%%%%%%%%%%%%%%%%%%%%%%%%%%%%%%
%Creates the aau titlepage
\newcommand{\aautitlepage}[3]{%
  {
    %set up various length
    \ifx\titlepageleftcolumnwidth\undefined
      \newlength{\titlepageleftcolumnwidth}
      \newlength{\titlepagerightcolumnwidth}
    \fi
    \setlength{\titlepageleftcolumnwidth}{0.48\textwidth-\tabcolsep}
    \setlength{\titlepagerightcolumnwidth}{\textwidth-2\tabcolsep-\titlepageleftcolumnwidth}
    %create title page
    \thispagestyle{empty}
    \noindent%
    \begin{tabular}{@{}ll@{}}
      \parbox{\titlepageleftcolumnwidth}{
        \iflanguage{danish}{%
          \includegraphics[width=\titlepageleftcolumnwidth-5mm]{media/AAUgraphics/aau_logo_da}
        }{%
          \includegraphics[width=\titlepageleftcolumnwidth-5mm]{media/AAUgraphics/aau_logo_en}
        }
      } &
      \parbox{\titlepagerightcolumnwidth}{\raggedleft\sf\small
        #2
      }\bigskip\\
       #1 &
      \parbox[t]{\titlepagerightcolumnwidth}{%
      \textbf{Abstract:}\bigskip\par
        \fbox{\parbox{\titlepagerightcolumnwidth-2\fboxsep-2\fboxrule}{%
          #3
        }}
      }\\
    \end{tabular}
    \vfill \bigskip
    \iflanguage{danish}{\noindent{\footnotesize\emph{Rapportens indhold er frit tilgængeligt, men offentliggørelse\\ (med kildeangivelse) må kun ske efter aftale med forfatterne.}}
    }{\noindent{\footnotesize\emph{The content of this report is freely available, but publication\\ (with reference) may only be pursued due to agreement with the author.}}
    }
    \clearpage
  }
}

%Create english project info
\newcommand{\englishprojectinfo}[8]{%
  \parbox[t]{\titlepageleftcolumnwidth}{
    \textbf{Title:}\\ #1\bigskip\par
    \textbf{Theme:}\\ #2\bigskip\par
    \textbf{Project Period:}\\ #3\bigskip\par
    \textbf{Project Group:}\\ #4\bigskip\par
    \textbf{Participant(s):}\\ #5\bigskip\par
    \textbf{Supervisor(s):}\\ #6\bigskip\par
    \textbf{Copies:} #7\bigskip\par
    \textbf{Number of pages:} \pageref{myLastPage}\bigskip\par
    \textbf{Date of Completion:}\\ #8
  }
}

%Create danish project info
\newcommand{\danishprojectinfo}[8]{%
  \parbox[t]{\titlepageleftcolumnwidth}{
    \textbf{Titel:}\\ #1\bigskip\par
    \textbf{Tema:}\\ #2\bigskip\par
    \textbf{Projektperiode:}\\ #3\bigskip\par
    \textbf{Projektgruppe:}\\ #4\bigskip\par
    \textbf{Deltager(e):}\\ #5\bigskip\par
    \textbf{Vejleder(e):}\\ #6\bigskip\par
    \textbf{Oplagstal:} #7\bigskip\par
    \textbf{Sidetal:} \pageref{myLastPage}\bigskip\par
    \textbf{Afleveringsdato:}\\ #8
  }
}

%%%%%%%%%%%%%%%%%%%%%%%%%%%%%%%%%%%%%%%%%%%%%%%%
% An example environment
%%%%%%%%%%%%%%%%%%%%%%%%%%%%%%%%%%%%%%%%%%%%%%%%
\theoremheaderfont{\normalfont\bfseries}
\theorembodyfont{\normalfont}
\theoremstyle{break}
\def\theoremframecommand{{\color{gray!50}\vrule width 5pt \hspace{5pt}}}
\newshadedtheorem{exa}{Example}[chapter]
\newenvironment{example}[1]{%
		\begin{exa}[#1]
}{%
		\end{exa}
} % Macroer der bruges igennem rapporten
%%
%% Denne fil bruges til at sætte variabler som bruges på 
%%

\def\myForsideMedBillede{1} % Skal forsiden være med billede? 1 = ja, 0 = nej
\def\myForsideBillede{media/m1.jpg} % Formateres i forholdet(højde:bredde) 1:2
\def\myProjectTitle{RoboCup 2022: BackDragR}
\def\mySubtitle{Development of a LEGO MindStorms EV3-robot to win the annual RoboCup}
\def\myProjectTheme{LEGO MindStorms-robotics}
\def\mySemester{Fall Semester} % 'Fall Semester' eller 'Spring Semester'

%Tre nedenstående felter skal oversættes til dansk til den danske titlepage
\def\myProjectTitleDA{RoboCup 2022: BackDragR}
\def\myProjectThemeDA{LEGO MindStorms-robotics}
\def\mySemesterDA{Fall Semester} % 'Forårssemesteret' eller 'Efterårssemesteret'

\def\mySemesterYear{2022}
\def\myAuthors{Bjørn Drachmann, Frida Brockie Davidsen, Lærke Rantzau Sørensen, Mathias Ramlov, Morten Zink Stage} %Gruppemedlemmer på én linje sepereret af komma
\def\myAuthorsNL{% her skrives hvert navn på en ny linje med et linjeskift imellem hver med komma efter navnene
Bjørn Drachmann, 

Frida Brockie Davidsen, 

Lærke Rantzau Sørensen, 

Mathias Ramlov, 

Morten Zink Stage
}
\def\mySupervisors{%
Melisa Maria Lopez Lechuga
}
\def\myProgramme{Computerteknologi}
\def\myGroupNumber{151}
\def\myProjectNumber{P0-Project} % Her skal der stå hvilket projekt man er ved, fx P0 project eller Bachelor's Project
\def\myColophon{%
Here you can write something about which tools and software you have used for typesetting the document, running simulations and creating figures. If you do not know what to write, either leave this page blank or have a look at the colophon in some of your books.
}
\def\myDepartment{Department of Electronic Systems}
\def\myDepartmentDA{Institut for Elektroniske Systemer}
\def\myDepartmentWebsite{https://www.es.aau.dk}% Indlæs opsætning fra documentSetup
% ----- Her defineres forkortelser - de kaldes med \ac{example}

\DeclareAcronym{eksempel}{
    short =     {EKS},
    long =      {Eksempel},
    long-post = { (denne tekst skrives ved siden af forkortelsen første gang den bruges)},
    extra =     {Denne tekst vises i listen over forkortelser}
}


% ----- Her defineres begreber - de kaldes med \gls{example}

\newglossaryentry{eksempel}{
    name = {eksempel},
    description = {En eksemplificering af noget}
}


\makeglossaries % Indlæs definitioner på begreber og forkortelser

% Ret denne i documentSetup - Fjerner tomme sider mellem kapitler osv i hele dokumentet
\ifthenelse{\myPageOptimizeForPrint=1}{}{\let\cleardoublepage\clearpage}
% Ret denne i documentSetup - Fjerner Indrykning efter nyt afsnit
\ifthenelse{\myParagraphIndentDisable=1}{\setlength\parindent{0pt}}{\setlength\parindent{\myParagraphIndentLength}}


% Hyperlinks / PDF metadata
% http://en.wikibooks.org/wiki/LaTeX/Hyperlinks
%%%%%%%%%%%%%%%%%%%%%%%%%%%%%%%%%%%%%%%%%%%%%%%%
% Enable hyperlinks and insert metadata into the 
% pdf file.
\usepackage[pdftex]{hyperref}
\hypersetup{%
	plainpages=false,%
	pdfauthor={\myAuthors},%
    pdftitle={\myProjectTitle},%
    pdfsubject={\myProjectTheme},%
	bookmarksnumbered=true,%
	colorlinks=false,%
	citecolor=black,%
	filecolor=black,%
	linkcolor=black,% you should probably change this to black before printing
	urlcolor=black,%
	pdfstartview=FitH%
}

\begin{document}

\pagestyle{empty} %disable headers and footers
\pagenumbering{roman} %use roman page numbering in the frontmatter

% ToDo-oversigt - udkommenteres før at dokumentet afleveres
\listoftodos

\ifthenelse{\myForsideMedBillede=1}{%%
%% Denne fil bruges til at sætte variabler som bruges på 
%%

\def\myForsideMedBillede{1} % Skal forsiden være med billede? 1 = ja, 0 = nej
\def\myForsideBillede{media/m1.jpg} % Formateres i forholdet(højde:bredde) 1:2
\def\myProjectTitle{RoboCup 2022: BackDragR}
\def\mySubtitle{Development of a LEGO MindStorms EV3-robot to win the annual RoboCup}
\def\myProjectTheme{LEGO MindStorms-robotics}
\def\mySemester{Fall Semester} % 'Fall Semester' eller 'Spring Semester'

%Tre nedenstående felter skal oversættes til dansk til den danske titlepage
\def\myProjectTitleDA{RoboCup 2022: BackDragR}
\def\myProjectThemeDA{LEGO MindStorms-robotics}
\def\mySemesterDA{Fall Semester} % 'Forårssemesteret' eller 'Efterårssemesteret'

\def\mySemesterYear{2022}
\def\myAuthors{Bjørn Drachmann, Frida Brockie Davidsen, Lærke Rantzau Sørensen, Mathias Ramlov, Morten Zink Stage} %Gruppemedlemmer på én linje sepereret af komma
\def\myAuthorsNL{% her skrives hvert navn på en ny linje med et linjeskift imellem hver med komma efter navnene
Bjørn Drachmann, 

Frida Brockie Davidsen, 

Lærke Rantzau Sørensen, 

Mathias Ramlov, 

Morten Zink Stage
}
\def\mySupervisors{%
Melisa Maria Lopez Lechuga
}
\def\myProgramme{Computerteknologi}
\def\myGroupNumber{151}
\def\myProjectNumber{P0-Project} % Her skal der stå hvilket projekt man er ved, fx P0 project eller Bachelor's Project
\def\myColophon{%
Here you can write something about which tools and software you have used for typesetting the document, running simulations and creating figures. If you do not know what to write, either leave this page blank or have a look at the colophon in some of your books.
}
\def\myDepartment{Department of Electronic Systems}
\def\myDepartmentDA{Institut for Elektroniske Systemer}
\def\myDepartmentWebsite{https://www.es.aau.dk}% Indlæs opsætning fra DokumentSetup
\pdfbookmark[0]{Front page}{label:frontpage}%
\begin{titlepage}
\newgeometry{top=0cm,bottom=1.2cm,right=0cm,left=0cm}
  \backgroundsetup{
   scale=1.1,
   angle=0,
   opacity=1.0,  %% adjust
   contents={\includegraphics[width=\paperwidth,height=\paperheight]{media/AAUgraphics/aau_waves}}
    }
  \begin{center} %%please do not change the height or width of the frontpage image
    \includegraphics[totalheight=0.5\paperwidth,width=1\paperwidth]{\myForsideBillede}
  \end{center}
	\vspace*{-0.96cm}
  {\noindent\color{aaublue}\fboxsep0pt\colorbox{white}{\begin{tabular}{@{}p{\paperwidth}@{}}
    \centerline{
    \begin{minipage}{0.85\textwidth}
        \bigskip
				\bigskip
        \centering
        \Huge{\textbf{\myProjectTitle}}
    \end{minipage}
    }
	\centerline{
	\begin{minipage}{0.9\textwidth}
        \bigskip
        \centering
        \Large{\mySubtitle}
    \end{minipage}
    }
	\centerline{
	\begin{minipage}{0.8\textwidth}
        \bigskip
        \centering
        {\Large
    \myAuthors{}
        }
    \end{minipage}
    }
    \centerline{
    \begin{minipage}{0.9\textwidth}
        \bigskip
        \centering
    \large{\myProgramme{}, \myGroupNumber{}, \mySemester{} \mySemesterYear{}}
    \end{minipage}
    }
    \centerline{
    \begin{minipage}{0.9\textwidth}
        \bigskip
        \centering
        \Large{\myProjectNumber}
        \smallskip
    \end{minipage}
    }
  \end{tabular}}}
  \vfill
  \begin{figure}[!b]
	\centering
    \includegraphics[width=0.2\paperwidth]{media/AAUgraphics/aau_logo_circle_en}
  \end{figure}
\end{titlepage}
\restoregeometry}{%%
%% Denne fil bruges til at sætte variabler som bruges på 
%%

\def\myForsideMedBillede{1} % Skal forsiden være med billede? 1 = ja, 0 = nej
\def\myForsideBillede{media/m1.jpg} % Formateres i forholdet(højde:bredde) 1:2
\def\myProjectTitle{RoboCup 2022: BackDragR}
\def\mySubtitle{Development of a LEGO MindStorms EV3-robot to win the annual RoboCup}
\def\myProjectTheme{LEGO MindStorms-robotics}
\def\mySemester{Fall Semester} % 'Fall Semester' eller 'Spring Semester'

%Tre nedenstående felter skal oversættes til dansk til den danske titlepage
\def\myProjectTitleDA{RoboCup 2022: BackDragR}
\def\myProjectThemeDA{LEGO MindStorms-robotics}
\def\mySemesterDA{Fall Semester} % 'Forårssemesteret' eller 'Efterårssemesteret'

\def\mySemesterYear{2022}
\def\myAuthors{Bjørn Drachmann, Frida Brockie Davidsen, Lærke Rantzau Sørensen, Mathias Ramlov, Morten Zink Stage} %Gruppemedlemmer på én linje sepereret af komma
\def\myAuthorsNL{% her skrives hvert navn på en ny linje med et linjeskift imellem hver med komma efter navnene
Bjørn Drachmann, 

Frida Brockie Davidsen, 

Lærke Rantzau Sørensen, 

Mathias Ramlov, 

Morten Zink Stage
}
\def\mySupervisors{%
Melisa Maria Lopez Lechuga
}
\def\myProgramme{Computerteknologi}
\def\myGroupNumber{151}
\def\myProjectNumber{P0-Project} % Her skal der stå hvilket projekt man er ved, fx P0 project eller Bachelor's Project
\def\myColophon{%
Here you can write something about which tools and software you have used for typesetting the document, running simulations and creating figures. If you do not know what to write, either leave this page blank or have a look at the colophon in some of your books.
}
\def\myDepartment{Department of Electronic Systems}
\def\myDepartmentDA{Institut for Elektroniske Systemer}
\def\myDepartmentWebsite{https://www.es.aau.dk}% Indlæs opsætning fra DokumentSetup
\pdfbookmark[0]{Front page}{label:frontpage}%
\begin{titlepage}
\vspace*{\fill}
    \backgroundsetup{
    scale=1.1,
    angle=0,
    opacity=1.0,  %% adjust
    contents={\includegraphics[width=\paperwidth,height=\paperheight]{media/AAUgraphics/aau_waves}}
    }
  \addtolength{\hoffset}{0.5\evensidemargin-0.5\oddsidemargin} %set equal margins on the frontpage - remove this line if you want default margins
  \noindent%
  {\color{white}\fboxsep7pt\colorbox{aaublue}{\begin{tabular}{@{}p{\textwidth}@{}}
    \begin{center}
    \Huge{\textbf{\myProjectTitle}}
    \end{center}
    \begin{center}
      \Large{\mySubtitle}
    \end{center}
    \vspace{0.15cm}
   \begin{center}
    \Large{\myAuthors}
    \vspace{0.15cm}
    \large{\myProgramme{}, \myGroupNumber{}, \mySemester{} \mySemesterYear{}}
   \end{center}
   \vspace{0.15cm}
%% Comment this section in if you are doing Bachelor or Master Project   
   \begin{center}
    \Large{\myProjectNumber}
   \end{center}
  \end{tabular}}}
  \vfill
  \begin{center}
    \includegraphics[width=0.2\paperwidth]{media/AAUgraphics/aau_logo_circle_en}% comment this line in for English version
  \end{center}
\end{titlepage}
\clearpage
}
%%
%% Denne fil bruges til at sætte variabler som bruges på 
%%

\def\myForsideMedBillede{1} % Skal forsiden være med billede? 1 = ja, 0 = nej
\def\myForsideBillede{media/m1.jpg} % Formateres i forholdet(højde:bredde) 1:2
\def\myProjectTitle{RoboCup 2022: BackDragR}
\def\mySubtitle{Development of a LEGO MindStorms EV3-robot to win the annual RoboCup}
\def\myProjectTheme{LEGO MindStorms-robotics}
\def\mySemester{Fall Semester} % 'Fall Semester' eller 'Spring Semester'

%Tre nedenstående felter skal oversættes til dansk til den danske titlepage
\def\myProjectTitleDA{RoboCup 2022: BackDragR}
\def\myProjectThemeDA{LEGO MindStorms-robotics}
\def\mySemesterDA{Fall Semester} % 'Forårssemesteret' eller 'Efterårssemesteret'

\def\mySemesterYear{2022}
\def\myAuthors{Bjørn Drachmann, Frida Brockie Davidsen, Lærke Rantzau Sørensen, Mathias Ramlov, Morten Zink Stage} %Gruppemedlemmer på én linje sepereret af komma
\def\myAuthorsNL{% her skrives hvert navn på en ny linje med et linjeskift imellem hver med komma efter navnene
Bjørn Drachmann, 

Frida Brockie Davidsen, 

Lærke Rantzau Sørensen, 

Mathias Ramlov, 

Morten Zink Stage
}
\def\mySupervisors{%
Melisa Maria Lopez Lechuga
}
\def\myProgramme{Computerteknologi}
\def\myGroupNumber{151}
\def\myProjectNumber{P0-Project} % Her skal der stå hvilket projekt man er ved, fx P0 project eller Bachelor's Project
\def\myColophon{%
Here you can write something about which tools and software you have used for typesetting the document, running simulations and creating figures. If you do not know what to write, either leave this page blank or have a look at the colophon in some of your books.
}
\def\myDepartment{Department of Electronic Systems}
\def\myDepartmentDA{Institut for Elektroniske Systemer}
\def\myDepartmentWebsite{https://www.es.aau.dk}% Indlæs opsætning fra DokumentSetup
\def\myAbstractEN{%
% Engelsk Abstract:
This is where we write out abstract for the project
}

\def\myAbstractDA{%
% Dansk resumé:
Det er her vi skriver vores resumé til projektet1-frontpageImage
}% Indlæs opsætning fra DokumentSetup
\pdfbookmark[0]{English title page}{label:titlepage_en}
\aautitlepage{%
  \englishprojectinfo{\myProjectTitle}{\myProjectTheme}{\mySemester{} \mySemesterYear}{\myGroupNumber}{\myAuthorsNL}{\mySupervisors}{%
    1 % number of printed copies
  }{%
    \today % date of completion
  }%
}{%department and address
  \textbf{\myProgramme}\\
  \myDepartment\\
  Aalborg University\\
  \href{\myDepartmentWebsite}{\myDepartmentWebsite}
}{\myAbstractEN}
\cleardoublepage

{\selectlanguage{danish}
\pdfbookmark[0]{Danish title page}{label:titlepage_da}
\aautitlepage{%
  \danishprojectinfo{\myProjectTitleDA}{\myProjectThemeDA}{\mySemesterDA{} \mySemesterYear}{\myGroupNumber}{\myAuthorsNL}{\mySupervisors}{%
    1 % number of printed copies
  }{%
    \today % date of completion
  }%
}{%department and address
  \textbf{\myProgramme}\\
  \myDepartmentDA\\
  Aalborg Universitet\\
  \href{\myDepartmentWebsite}{\myDepartmentWebsite}
}{\myAbstractDA}}
       % Titelsider på engelsk og på dansk
%%
%% Denne fil bruges til at sætte variabler som bruges på 
%%

\def\myForsideMedBillede{1} % Skal forsiden være med billede? 1 = ja, 0 = nej
\def\myForsideBillede{media/m1.jpg} % Formateres i forholdet(højde:bredde) 1:2
\def\myProjectTitle{RoboCup 2022: BackDragR}
\def\mySubtitle{Development of a LEGO MindStorms EV3-robot to win the annual RoboCup}
\def\myProjectTheme{LEGO MindStorms-robotics}
\def\mySemester{Fall Semester} % 'Fall Semester' eller 'Spring Semester'

%Tre nedenstående felter skal oversættes til dansk til den danske titlepage
\def\myProjectTitleDA{RoboCup 2022: BackDragR}
\def\myProjectThemeDA{LEGO MindStorms-robotics}
\def\mySemesterDA{Fall Semester} % 'Forårssemesteret' eller 'Efterårssemesteret'

\def\mySemesterYear{2022}
\def\myAuthors{Bjørn Drachmann, Frida Brockie Davidsen, Lærke Rantzau Sørensen, Mathias Ramlov, Morten Zink Stage} %Gruppemedlemmer på én linje sepereret af komma
\def\myAuthorsNL{% her skrives hvert navn på en ny linje med et linjeskift imellem hver med komma efter navnene
Bjørn Drachmann, 

Frida Brockie Davidsen, 

Lærke Rantzau Sørensen, 

Mathias Ramlov, 

Morten Zink Stage
}
\def\mySupervisors{%
Melisa Maria Lopez Lechuga
}
\def\myProgramme{Computerteknologi}
\def\myGroupNumber{151}
\def\myProjectNumber{P0-Project} % Her skal der stå hvilket projekt man er ved, fx P0 project eller Bachelor's Project
\def\myColophon{%
Here you can write something about which tools and software you have used for typesetting the document, running simulations and creating figures. If you do not know what to write, either leave this page blank or have a look at the colophon in some of your books.
}
\def\myDepartment{Department of Electronic Systems}
\def\myDepartmentDA{Institut for Elektroniske Systemer}
\def\myDepartmentWebsite{https://www.es.aau.dk}% Indlæs opsætning fra DocumentSetup
\thispagestyle{empty}
{\small
\strut\vfill % push the content to the bottom of the page
\noindent Copyright \copyright{}\mySemesterYear{} Aalborg University\\
\par
\noindent This report is based on a LaTeX template by Morten Zink Stage\\
\url{https://github.com/Peasniped/AAU-LateX-Report-template}.\\
Thanks to Jesper Kjær Nielsen for the work the template was initially adapted from.\par
\vspace{0.2cm}
\noindent \myColophon
}

\clearpage

\pdfbookmark[0]{Contents}{label:contents}
\pagestyle{fancy} %enable headers and footers again
\tableofcontents
\setlength{\parskip}{\myParagraphSkipLength} % Ret i documentSetup - Afstand ml. paragraffer
%%
%% Denne fil bruges til at sætte variabler som bruges på 
%%

\def\myForsideMedBillede{1} % Skal forsiden være med billede? 1 = ja, 0 = nej
\def\myForsideBillede{media/m1.jpg} % Formateres i forholdet(højde:bredde) 1:2
\def\myProjectTitle{RoboCup 2022: BackDragR}
\def\mySubtitle{Development of a LEGO MindStorms EV3-robot to win the annual RoboCup}
\def\myProjectTheme{LEGO MindStorms-robotics}
\def\mySemester{Fall Semester} % 'Fall Semester' eller 'Spring Semester'

%Tre nedenstående felter skal oversættes til dansk til den danske titlepage
\def\myProjectTitleDA{RoboCup 2022: BackDragR}
\def\myProjectThemeDA{LEGO MindStorms-robotics}
\def\mySemesterDA{Fall Semester} % 'Forårssemesteret' eller 'Efterårssemesteret'

\def\mySemesterYear{2022}
\def\myAuthors{Bjørn Drachmann, Frida Brockie Davidsen, Lærke Rantzau Sørensen, Mathias Ramlov, Morten Zink Stage} %Gruppemedlemmer på én linje sepereret af komma
\def\myAuthorsNL{% her skrives hvert navn på en ny linje med et linjeskift imellem hver med komma efter navnene
Bjørn Drachmann, 

Frida Brockie Davidsen, 

Lærke Rantzau Sørensen, 

Mathias Ramlov, 

Morten Zink Stage
}
\def\mySupervisors{%
Melisa Maria Lopez Lechuga
}
\def\myProgramme{Computerteknologi}
\def\myGroupNumber{151}
\def\myProjectNumber{P0-Project} % Her skal der stå hvilket projekt man er ved, fx P0 project eller Bachelor's Project
\def\myColophon{%
Here you can write something about which tools and software you have used for typesetting the document, running simulations and creating figures. If you do not know what to write, either leave this page blank or have a look at the colophon in some of your books.
}
\def\myDepartment{Department of Electronic Systems}
\def\myDepartmentDA{Institut for Elektroniske Systemer}
\def\myDepartmentWebsite{https://www.es.aau.dk}% Indlæs opsætning fra DocumentSetup

\chapter*{Preface\markboth{Preface}{Preface}}\label{ch:preface}
\addcontentsline{toc}{chapter}{Preface}

\noindent
Here is the preface. You should put your signatures at the end of the preface.\\
\par
\noindent
This report was written by\mySemesterNumberOrdinal{} semester students (group \myGroupNumber{}), studying \myProgramme{} at Aalborg university during the \mySemester{} of \mySemesterYear{}.\\
\noindent
We would like to \myThankOurSupervisors{} for their help and guidance throughout the project.\\

\par
\noindent
Reading Guide:
For figures, listings and tables the first number is the chapter and the last numbers are the sequential number of the order they are presented in.\\ 
Source references use a numbered system [Number] with the sources listed in the bibliography section in numerated order at the end of the report.


\vspace{\baselineskip}\hfill Aalborg University, \today

\ifthenelse{\myAuthorsNUM=1}{
   \vfill\noindent
    \begin{center}
    \begin{minipage}[b]{0.45\textwidth}
     \centering
     \rule{\textwidth}{0.5pt}
      \myAuthorOneName\\
     {\footnotesize <\myAuthorOneEmail{}>}
    \end{minipage}
    \end{center}

}{ \ifthenelse{\myAuthorsNUM=2}{
    \vfill\noindent
    \begin{minipage}[b]{0.45\textwidth}
     \centering
     \rule{\textwidth}{0.5pt}\\
      \myAuthorOneName\\
     {\footnotesize <\myAuthorOneEmail{}>}
    \end{minipage}
    \hfill
    \begin{minipage}[b]{0.45\textwidth}
     \centering
     \rule{\textwidth}{0.5pt}\\
      \myAuthorTwoName\\
     {\footnotesize <\myAuthorTwoEmail{}>}
    \end{minipage}

}{ \ifthenelse{\myAuthorsNUM=3}{
    \vfill\noindent
    \begin{minipage}[b]{0.45\textwidth}
     \centering
     \rule{\textwidth}{0.5pt}\\
      \myAuthorOneName\\
     {\footnotesize <\myAuthorOneEmail{}>}
    \end{minipage}
    \hfill
    \begin{minipage}[b]{0.45\textwidth}
     \centering
     \rule{\textwidth}{0.5pt}\\
      \myAuthorTwoName\\
     {\footnotesize <\myAuthorTwoEmail{}>}
    \end{minipage}
    
    \vspace{3\baselineskip}
    
    \begin{center}
    \begin{minipage}[b]{0.45\textwidth}
     \centering
     \rule{\textwidth}{0.5pt}
      \myAuthorThreeName\\
     {\footnotesize <\myAuthorThreeEmail{}>}
    \end{minipage}
    \end{center}

}{ \ifthenelse{\myAuthorsNUM=4}{
    \vfill\noindent
    \begin{minipage}[b]{0.45\textwidth}
     \centering
     \rule{\textwidth}{0.5pt}\\
      \myAuthorOneName\\
     {\footnotesize <\myAuthorOneEmail{}>}
    \end{minipage}
    \hfill
    \begin{minipage}[b]{0.45\textwidth}
     \centering
     \rule{\textwidth}{0.5pt}\\
      \myAuthorTwoName\\
     {\footnotesize <\myAuthorTwoEmail{}>}
    \end{minipage}
    
    \vspace{2\baselineskip}
    
    \noindent
    \begin{minipage}[b]{0.45\textwidth}
     \centering
     \rule{\textwidth}{0.5pt}\\
      \myAuthorThreeName\\
     {\footnotesize <\myAuthorThreeEmail{}>}
    \end{minipage}
    \hfill
    \begin{minipage}[b]{0.45\textwidth}
     \centering
     \rule{\textwidth}{0.5pt}\\
      \myAuthorFourName\\
     {\footnotesize <\myAuthorFourEmail{}>}
    \end{minipage}

}{ \ifthenelse{\myAuthorsNUM=5}{
    \vfill\noindent
    \begin{minipage}[b]{0.45\textwidth}
     \centering
     \rule{\textwidth}{0.5pt}\\
      \myAuthorOneName\\
     {\footnotesize <\myAuthorOneEmail{}}
    \end{minipage}
    \vspace{3\baselineskip}
    \hfill
    \begin{minipage}[b]{0.45\textwidth}
     \centering
     \rule{\textwidth}{0.5pt}\\
      \myAuthorTwoName\\
     {\footnotesize <\myAuthorTwoEmail{}>}
    \end{minipage}
    \begin{minipage}[b]{0.45\textwidth}
     \centering
     \rule{\textwidth}{0.5pt}\\
      \myAuthorThreeName\\
     {\footnotesize <\myAuthorThreeEmail{}>}
    \end{minipage}
    \hfill
    \begin{minipage}[b]{0.45\textwidth}
     \centering
     \rule{\textwidth}{0.5pt}\\
      \myAuthorFourName\\
     {\footnotesize <\myAuthorFourEmail{}>}
    \end{minipage}
    \vspace{3\baselineskip}
    \begin{center}
    \begin{minipage}[b]{0.45\textwidth}
     \centering
     \rule{\textwidth}{0.5pt}
      \myAuthorFiveName\\
     {\footnotesize <\myAuthorFiveEmail{}>}
    \end{minipage}
    \end{center}

}{ \ifthenelse{\myAuthorsNUM=6}{
    \vfill\noindent
    \begin{minipage}[b]{0.45\textwidth}
     \centering
     \rule{\textwidth}{0.5pt}\\
      \myAuthorOneName\\
     {\footnotesize <\myAuthorOneEmail{}>}
    \end{minipage}
    \vspace{3\baselineskip}
    \hfill
    \begin{minipage}[b]{0.45\textwidth}
     \centering
     \rule{\textwidth}{0.5pt}\\
      \myAuthorTwoName\\
     {\footnotesize <\myAuthorTwoEmail{}>}
    \end{minipage}
    \begin{minipage}[b]{0.45\textwidth}
     \centering
     \rule{\textwidth}{0.5pt}\\
      \myAuthorThreeName\\
     {\footnotesize <\myAuthorThreeEmail{}>}
    \end{minipage}
    \vspace{3\baselineskip}
    \hfill
    \begin{minipage}[b]{0.45\textwidth}
     \centering
     \rule{\textwidth}{0.5pt}\\
      \myAuthorFourName\\
     {\footnotesize <\myAuthorFourEmail{}>}
    \end{minipage}
    \begin{minipage}[b]{0.45\textwidth}
     \centering
     \rule{\textwidth}{0.5pt}\\
      \myAuthorFiveName\\
     {\footnotesize <\myAuthorFiveEmail{}>}
    \end{minipage}
    \hfill
    \begin{minipage}[b]{0.45\textwidth}
     \centering
     \rule{\textwidth}{0.5pt}\\
      \myAuthorSixName\\
     {\footnotesize <\myAuthorSixEmail{}>}
    \end{minipage}
}{ !!! fejl: myAuthorsNUM skal være et tal mellem 1-6, myAuthorsNUM er \myAuthorsNUM !!! }}}}}}

% Printer Forklaring til forkortelser og ordbog
% acro v2 syntax:
\printacronyms[heading={chapter*}]
\addcontentsline{toc}{chapter}{Acronyms} 
\printglossaries
\pagebreak

\pagenumbering{arabic} %use Arabic page numbering in the main matter
\chapter{Introduction}\label{ch:introduction}
Here is the introduction. The next chapter is chapter. ~\ref{ch:ch2label}.


a new paragraph


\section{Examples}
You can also have examples in your document such as in example~\ref{ex:simple_example}.
\begin{example}{An Example of an Example}
  \label{ex:simple_example}
  Here is an example with some math
  \begin{equation}
    0 = \exp(i\pi)+1\ .
  \end{equation}
  You can adjust the colour and the line width in the {\tt macros.tex} file.
\end{example}

\begin{table}
    \noindent\begin{tabular}{|l|c|}
        \rowcolor{aaublue}
        {\color[HTML]{FFFFFF} one} & {\color[HTML]{FFFFFF} two} \\
        three & four \\
        five & six \\ \hline
        
    \end{tabular}
    \caption{A table with some things in}
    \label{tab:table}
\end{table}

\section{How Does Sections, Subsections, and Subsections Look?}
Well, like this
\subsection{This is a Subsection}
and this
\subsubsection{This is a Subsubsection}
and this.

\paragraph{A Paragraph}
You can also use paragraph titles which look like this.

This is a reference to \myRef{Table}{tab:table}
\newline
This is a reference with a page number to \myRefPage{Table}{tab:table}

\subparagraph{A Subparagraph} Moreover, you can also use subparagraph titles which look like this\todo{Is it possible to add a subsubparagraph?}. They have a small indentation as opposed to the paragraph titles.

\todo[inline,color=green]{I think that a summary of this exciting chapter should be added.}

\setlength{\headheight}{14pt}
\chapter{Chapter 2 name}\label{ch:ch2label}
\ifthenelse{\myExpandAcronymsEveryChapter=1}{\acresetall}{} % Nulstiller alle akronymers "er brugt"-tilstand. Indstilles i documentSetup.

Here is chapter 2. If you want to leearn \todo{I think this word is mispelled} more about \LaTeXe{}, have a look at \cite{Madsen2010}, \cite{Oetiker2010} and \cite{Mittelbach2005}.
\missingfigure{We need a figure right here!}
\chapter{Requirements}\label{ch:requirements}
\ifthenelse{\myExpandAcronymsEveryChapter=1}{\acresetall}{} % Nulstiller alle akronymers "er brugt"-tilstand. Indstilles i documentSetup.
\chapter{Design}\label{ch:design}
\ifthenelse{\myExpandAcronymsEveryChapter=1}{\acresetall}{} % Nulstiller alle akronymers "er brugt"-tilstand. Indstilles i documentSetup.
\chapter{Implementation}\label{ch:implementation}
\ifthenelse{\myExpandAcronymsEveryChapter=1}{\acresetall}{} % Nulstiller alle akronymers "er brugt"-tilstand. Indstilles i documentSetup.
\chapter{Test and Validation}\label{ch:test and validation}
\ifthenelse{\myExpandAcronymsEveryChapter=1}{\acresetall}{} % Nulstiller alle akronymers "er brugt"-tilstand. Indstilles i documentSetup.
\input{sections/6.7- discussion}
\chapter{Conclusion}\label{ch:conclusion}
\ifthenelse{\myExpandAcronymsEveryChapter=1}{\acresetall}{} % Nulstiller alle akronymers "er brugt"-tilstand. Indstilles i documentSetup.

There are probably still some bugs in the template. If you should find one, then please submit it on \url{https://github.com/Peasniped/AAU-LateX-Report-template/issues}.

\printbibliography[heading=bibintoc, title=Bibliography]
\label{bib:bibliography}

\label{myLastPage}
\clearpage

\appendix
%NB: Kommandoen '\myAppendixPageNumbering' kaldes før indsættelsen af et appendix(bilag). Kommandoen nulstiller sidetallet og parameteren 'A-' definerer prefix på sidetallene, så side 11 i bilag A formateres som side 'A-11'. Kommandoen skal kaldes på ny før hvert bilag(fx, bilag A, B, C etc.).
\myAppendixPageNumbering{A-}
\input{sections/8.a- appendix A.tex}
\clearpage

\myAppendixPageNumbering{B-}
\chapter{Kodeeksempler}\label{ch:appendix-kodeeksempler}

\section{Enumerate / Itemize}
Enumerate bruges til at lave en liste, fx med tal, bogstaver eller romertal. Itemize bruges til bullets.

\begin{lstlisting}[language=Tex, caption=Her vises enumerate med forskellige formatering af tallet]
 \begin{enumerate}[label=\textbf{\arabic*: }]
    \item This is the first item
    \item This is the second item
\end{enumerate}

 \begin{enumerate}[label=\textit{\roman*.  }]   % roman = i ii iii, Roman = I II III
    \item This is the third item
    \item This is the fourth item
\end{enumerate}

 \begin{enumerate}[label=\Alph*) ]    % alph = a b c, Alph = A B C
    \item This is the fifth item
    \item This is the sixth item
\end{enumerate}

\begin{itemize}[itemsep=1pt]    % itemsep = afstand ml punkter
    \item This is the final item
\end{itemize}
\end{lstlisting}

\subsection{Eksempel:}

 \begin{enumerate}[label=\textbf{\arabic*: }]
    \item This is the first item
    \item This is the second item
\end{enumerate}

 \begin{enumerate}[label=\textit{\roman*.  }]   % roman = i ii iii, Roman = I II III
    \item This is the third item
    \item This is the fourth item
\end{enumerate}

 \begin{enumerate}[label=\Alph*) ]    % alph = a b c, Alph = A B C
    \item This is the fifth item
    \item This is the sixth item
\end{enumerate}

\begin{itemize}[itemsep=1pt]    % itemsep = afstand ml punkter
    \item This is the final item
\end{itemize}

\pagebreak

\section{wrapTable / wrapFigure}
Wraptable/figure bruges til at få teksten til at gå rundt om billedet. Ligesom i Word, hvor man kan vælge at wrape teksten rundt om et billede.
De to environments har samme formatering, men wrapTable gør så elementet referes som tabel, hvis man bruger en label, og wrapFigure refererer til . 

\begin{lstlisting}[language=Tex, caption=Her vises brug af wrapTable]
    \lipsum[4]    % Dummytekst
    \begin{wrapfigure}{r}{8cm} % {r} betyder hoejrejusteret, {8cm} er bredden af containeren
    
        \centering  % Centrerer billedet i wrap-containeren
        \includegraphics[width=0.95\linewidth]{media/AAUgraphics/frontpageImage.jpg} % width er billedets bredde, linewidth refererer til bredden af wrap-containeren
        
        \caption{SPAAAAAAAAAAAAAAAAACEEEEEEEE!}
        \label{fig:wraptable}
        
    \end{wrapfigure}
    
    \lipsum[4-5]    % Dummytekst
\end{lstlisting}

\subsection{Eksempel:}
\lipsum[4]    % Dummytekst
\begin{wrapfigure}{r}{8cm} % {r} betyder højrejusteret, {8cm} er bredden af containeren
    \centering
    \includegraphics[width=0.95\linewidth]{media/AAUgraphics/frontpageImage.jpg} 
    % width er billedets bredde, linewidth refererer til bredden af containeren, så 0.95*linewidth giver lidt margin om billedet. 
    \caption{SPAAAAAAAAAAAAAAAAACEEEEEEEE!}
    \label{fig:wraptable}
\end{wrapfigure}
\vspace{-0cm} % nogle gange er der whitespace efter figuren. Det fjernes med fx -1cm vspace
\lipsum[4-5]    % Dummytekst

\pagebreak

\section{Minipages - To billeder ved siden af hinanden}
Hvis vi gerne vil have to billeder ved siden af hinanden kan vi dele siden op i to minipages.

\begin{lstlisting}[language=Tex, caption=Her vises brug af wrapTable]
\lipsum[64]
\par
\vspace{1cm}
\noindent % vigtigt at have med
\begin{minipage}{0.5\textwidth} % hver minipage er 0.5 sidebredde i bred
    \centering
    \includegraphics[width=0.8\linewidth]{media/AAUgraphics/aau_logo_circle_en.pdf} 
    \captionof{figure}{AAUs cirkel-logo paa engelsk} % NB: captionof ikke caption
    \label{fig:aauLogoEN}
\end{minipage}%
%
\begin{minipage}{0.5\textwidth}
    \centering
    \includegraphics[width=0.8\linewidth]{media/AAUgraphics/aau_logo_circle_da.pdf} 
    \captionof{figure}{AAUs cirkel-logo paa engelsk} % NB: captionof ikke caption
    \label{fig:aauLogoDA}
\end{minipage}
\end{lstlisting}

\noindent
Note: Vi bruger kommandoen captionof{figure}{caption skrives her} i stedet for almindeligvist at bruge caption{caption skrives her}, da vi ikke er inde i et korrekt "float environment". Det er fordi vi er inde i et minipage-environment som ikke definerer om det er en figur eller en tabel der er tale om.

\subsection{Eksempel:}
\lipsum[64]
\par
\vspace{1cm}
\noindent % vigtigt at have med
\begin{minipage}{0.5\textwidth} % hver minipage er 0.5 sidebredde i bred
    \centering
    \includegraphics[width=0.65\linewidth]{media/AAUgraphics/aau_logo_circle_en.pdf} 
    \captionof{figure}{AAUs cirkel-logo på engelsk} % bemærk \captionof ikke \caption
    \label{fig:aauLogoEN}
\end{minipage}%
%
\begin{minipage}{0.5\textwidth}
    \centering
    \includegraphics[width=0.65\linewidth]{media/AAUgraphics/aau_logo_circle_da.pdf} 
    \captionof{figure}{AAUs cirkel-logo på engelsk} % bemærk \captionof ikke \caption
    \label{fig:aauLogoDA}
\end{minipage}





\end{document}
