\chapter{Introduction}\label{ch:introduction}
Here is the introduction. The next chapter is chapter. ~\ref{ch:ch2label}.


a new paragraph


\section{Examples}
You can also have examples in your document such as in example~\ref{ex:simple_example}.
\begin{example}{An Example of an Example}
  \label{ex:simple_example}
  Here is an example with some math
  \begin{equation}
    0 = \exp(i\pi)+1\ .
  \end{equation}
  You can adjust the colour and the line width in the {\tt macros.tex} file.
  \vspace{3cm}
\end{example}

\begin{table}[h!]
    \centering
    \begin{tblr}{
        colspec = {l l l},
        vlines = {},    % Vertikale linjer
    	hlines = {},    % Horisontale linjer
    	rows = {bg = aaublue!15},   % farver alle rækker blå med 15% opacity
    	row{odd} = {bg = aaublue!30},   % farver hver anden række blå med 30% opacity (mørkere)
    	column{1} = {bg = aaublue!65, fg = white,halign = c},    % farver første kolonne blå/lilla og gør teksten hvid og centrerer tekst
    	row{1} = {bg = aaublue, fg = white},    % farver øverste række blå med hvid tekst
    	% merging cells
    	cell{2}{1} = {r = 2}{valign = m},   % merger celle 2,1 en række ned
    	cell{4}{2} = {c = 2}{halign = c},   % merger celle 4,2 en kolonne hen
        }
        \textbf{one} & \textbf{two} & \textbf{three}\\
        four         & five         & six\\
                     & seven        & eight\\
        nine         & ten          & \\
    \end{tblr}
    \caption{this is an example of a table(tabularray / tblr) with colors and merged cells}
    \label{tab:table}
\end{table}





\section{How Does Sections, Subsections, and Subsections Look?}
Well, like this
\subsection{This is a Subsection}
and this
\subsubsection{This is a Subsubsection}
and this.

\paragraph{A Paragraph}
You can also use paragraph titles which look like this.

This is a reference to \myRef{Table}{tab:table}
\newline
This is a reference with a page number to \myRefPage{Table}{tab:table}

\subparagraph{A Subparagraph} Moreover, you can also use subparagraph titles which look like this\todo{Is it possible to add a subsubparagraph?}. They have a small indentation as opposed to the paragraph titles.

\todo[inline,color=green]{I think that a summary of this exciting chapter should be added.}
