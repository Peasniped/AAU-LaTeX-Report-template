\chapter{Introduction}\label{ch:introduction}
Here is the introduction. The next chapter is chapter. ~\ref{ch:ch2label}.


a new paragraph


\section{Examples}
You can also have examples in your document such as in example~\ref{ex:simple_example}.
\begin{example}{An Example of an Example}
  \label{ex:simple_example}
  Here is an example with some math
  \begin{equation}
    0 = \exp(i\pi)+1\ .
  \end{equation}
  You can adjust the colour and the line width in the {\tt macros.tex} file.
\end{example}

\begin{center}
\noindent\begin{tabular}{|l|c|}
\rowcolor{aaublue}
{\color[HTML]{FFFFFF} one} & {\color[HTML]{FFFFFF} two} \\
three & four \\
five & six \\ \hline
\end{tabular}
\end{center}

\section{How Does Sections, Subsections, and Subsections Look?}
Well, like this
\subsection{This is a Subsection}
and this
\subsubsection{This is a Subsubsection}
and this.

\paragraph{A Paragraph}
You can also use paragraph titles which look like this.

\subparagraph{A Subparagraph} Moreover, you can also use subparagraph titles which look like this\todo{Is it possible to add a subsubparagraph?}. They have a small indentation as opposed to the paragraph titles.

\todo[inline,color=green]{I think that a summary of this exciting chapter should be added.}
