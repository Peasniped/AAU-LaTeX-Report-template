%%
%% Preamble er en slags side med indstillinger for hele rapporten
%%

%%%%%%%%%%%%%%%%%%%%%%%%%%%%%%%%%%%%%%%%%%%%%%%%
% Language, Encoding and Fonts
% http://en.wikibooks.org/wiki/LaTeX/Internationalization
%%%%%%%%%%%%%%%%%%%%%%%%%%%%%%%%%%%%%%%%%%%%%%%%
% Select encoding of your inputs. Depends on
% your operating system and its default input
% encoding. Typically, you should use
%   Linux  : utf8 (most modern Linux distributions)
%            latin1 
%   Windows: ansinew
%            latin1 (works in most cases)
%   Mac    : applemac
% Notice that you can manually change the input
% encoding of your files by selecting "save as"
% an select the desired input encoding. 
\usepackage[utf8]{inputenc}
\usepackage{ifthen}
% Make latex understand and use the typographic
% rules of the language used in the document.
\usepackage[danish,english]{babel}
% Use the palatino font
\usepackage[sc]{mathpazo}
\linespread{1.05}         % Palatino needs more leading (space between lines)
% Choose the font encoding
\usepackage[T1]{fontenc}

%%%%%%%%%%%%%%%%%%%%%%%%%%%%%%%%%%%%%%%%%%%%%%%%
% Graphics and Tables
% http://en.wikibooks.org/wiki/LaTeX/Importing_Graphics
% http://en.wikibooks.org/wiki/LaTeX/Tables
% http://en.wikibooks.org/wiki/LaTeX/Colors
%%%%%%%%%%%%%%%%%%%%%%%%%%%%%%%%%%%%%%%%%%%%%%%%

% load a colour package
\usepackage[table]{xcolor}
\definecolor{aaublue}{RGB}{33,26,82}% dark blue
\definecolor{aaublue}{RGB}{33,26,82}
\definecolor{white}{RGB}{255,255,255}
\definecolor{black}{RGB}{0,0,0}

% Indlæs pakke der laver bedre tabeller
\usepackage{float}
\usepackage{tabularray}
% How to use Tabularray:
% https://www.latex-tables.com/ressources/tabularray.html
\usepackage{tabularx}

%%% Indlæs pakke der gør det muligt at lave figurer/tabeller ved siden af teksten
% Ligesom i Word hvor man kan formatere billedet som 'Tæt'
\usepackage{wrapfig}
%% Brugseksempel:
% Skriv noget tekst her som skal gå rundt om billedet
% \begin{wrapfigure}{r}{7.5cm} % <-- wraptable for tabel
%   ## Billede her ##
% \end{wrapfigure} 

% The standard graphics inclusion package
\usepackage{graphicx}

% Set up how figure and table captions are displayed
\usepackage{caption}
\captionsetup{%
  font=footnotesize,% set font size to footnotesize
  labelfont=bf % bold label (e.g., Figure 3.2) font
}
% Make the standard latex tables look so much better
\usepackage{array,booktabs}
% Enable the use of frames around, e.g., theorems
% The framed package is used in the example environment
\usepackage{framed}

% Adds support for full page background picture
\usepackage{watermark}
%\usepackage[contents={},color=gray]{background} % <--- Denne linje laver everypage-warning

%%%%%%%%%%%%%%%%%%%%%%%%%%%%%%%%%%%%%%%%%%%%%%%%
% Mathematics
% http://en.wikibooks.org/wiki/LaTeX/Mathematics
%%%%%%%%%%%%%%%%%%%%%%%%%%%%%%%%%%%%%%%%%%%%%%%%
% Defines new environments such as equation,
% align and split 
\usepackage{amsmath}
% Adds new math symbols
\usepackage{amssymb}
% Use theorems in your document
% The ntheorem package is also used for the example environment
% When using thmmarks, amsmath must be an option as well. Otherwise \eqref doesn't work anymore.
\usepackage[framed,amsmath,thmmarks]{ntheorem}

%%%%%%%%%%%%%%%%%%%%%%%+%%%%%%%%%%%%%%%%%%%%%%%%%
% Page Layout
% http://en.wikibooks.org/wiki/LaTeX/Page_Layout
%%%%%%%%%%%%%%%%%%%%%%%%%%%%%%%%%%%%%%%%%%%%%%%%
% Change margins, papersize, etc of the document

% Marginer optimeret til bog
% \usepackage[inner=40mm, outer=25mm, top=20mm, bottom=30mm,]{geometry}

% Marginer optimeret til web
\usepackage[inner=20mm, outer=20mm, top=20mm, bottom=35mm,]{geometry}
  
% Modify how \chapter, \section, etc. look
% The titlesec package is very configureable
\usepackage{titlesec}

% Old chapter title format:
%\titleformat{\chapter}[display]{\normalfont\huge\bfseries}{\chaptertitlename: \thechapter}{20pt}{\Huge}
% New chapter title format:
\titleformat{\chapter}[hang]{\normalfont\huge\bfseries}{\chaptertitlename\ \thechapter:}{20pt}{\Huge}

\titleformat*{\section}{\normalfont\Large\bfseries}
\titleformat*{\subsection}{\normalfont\large\bfseries}
\titleformat*{\subsubsection}{\normalfont\normalsize\bfseries}
%\titleformat*{\paragraph}{\normalfont\normalsize\bfseries}
%\titleformat*{\subparagraph}{\normalfont\normalsize\bfseries}

% Clear empty pages between chapters
\let\origdoublepage\cleardoublepage
\newcommand{\clearemptydoublepage}{%
  \clearpage
  {\pagestyle{empty}\origdoublepage}%
}
\let\cleardoublepage\clearemptydoublepage

% Change the headers and footers
\usepackage{fancyhdr}
\pagestyle{fancy}
\fancyhf{} %delete everything
\renewcommand{\headrulewidth}{0pt} %remove the horizontal line in the header
\fancyhead[RE]{\small\nouppercase\leftmark} %even page - chapter title
\fancyhead[LO]{\small\nouppercase\rightmark} %uneven page - section title
\fancyhead[LE,RO]{\thepage} %page number on all pages
% Do not stretch the content of a page. Instead,
% insert white space at the bottom of the page
\raggedbottom
% Enable arithmetics with length. Useful when
% typesetting the layout.
\usepackage{calc}

% Package to use landscape-environment
\usepackage{pdflscape}

%%%%%%%%%%%%%%%%%%%%%%%%%%%%%%%%%%%%%%%%%%%%%%%%
% Bibliography
% http://en.wikibooks.org/wiki/LaTeX/Bibliography_Management
%%%%%%%%%%%%%%%%%%%%%%%%%%%%%%%%%%%%%%%%%%%%%%%%

\usepackage[backend=biber,
            style=numeric,
            sorting=none,
            bibencoding=utf8
            ]{biblatex}
\addbibresource{sections/7- bibliography.bib}
\usepackage{csquotes}

% Add bibliography and index to the table of
% contents
\usepackage[nottoc]{tocbibind}
\usepackage{comment}

%%%%%%%%%%%%%%%%%%%%%%%%%%%%%%%%%%%%%%%%%%%%%%%%
% Misc
%%%%%%%%%%%%%%%%%%%%%%%%%%%%%%%%%%%%%%%%%%%%%%%%

%%%% Acronyms og Glossary
% Aktiverer funktionalitet til at skrive og forklare forkortelser/begreber
% Begreber/forkortelser defineres i 5- glossary.tex, og bruges med \ac{} igennem teksten
\usepackage{acro}
\acsetup{use-id-as-short}
\usepackage[toc]{glossaries}
\usepackage{longtable}

%% Add todo notes in the margin of the document
\usepackage[
%  disable, %turn off todonotes
  colorinlistoftodos, %enable a coloured square in the list of todos
  textwidth=\marginparwidth, %set the width of the todonotes
  ]{todonotes}
  
%% Allows to enumerate with roman numbers
\usepackage{enumitem}
% Eksempel:
% \begin{enumerate}[label=\textbf{\arabic*: }]
    % \item item 1
    % \item item 2
    % \item item 3
% \end{enumerate}

%% Lipsum Pakken
% Pakken tilføjer kommando til at lave dummy-teksten 'lorem ipsum dolor sit amet'
\usepackage{lipsum}
% Eksempel:
% Man skriver \lipsum[x] og på x's plads skriver man de afsnit som man vil have. Pakken indeholder 150 sætninger, så tallet skal være imellem de to [1-150].
% \lipsum[1]        % Her får vi det første afsnit af dummy-teksten
% \lipsum[]         % Hvis man ikke angiver et tal får man afsnit 1 til 7 (ca 1½ sides tekst)
% \lipsum[24-26]    % her får man afsnitene 24 til 26 


%%%%%%%%%%%%%%%%%%%%%%%%%%%%%%%%%%%%%%%%%%%%%%%%
% Hyperlinks
% http://en.wikibooks.org/wiki/LaTeX/Hyperlinks
%%%%%%%%%%%%%%%%%%%%%%%%%%%%%%%%%%%%%%%%%%%%%%%%
% Enable hyperlinks and insert info into the pdf
% file. Hypperref should be loaded as one of the 
% last packages
\usepackage{hyperref}
\hypersetup{%
	%pdfpagelabels=true,%
	plainpages=false,%
	pdfauthor={Author(s)},%
	pdftitle={Title},%
	pdfsubject={Subject},%
	bookmarksnumbered=true,%
	colorlinks=false,%
	citecolor=black,%
	filecolor=black,%
	linkcolor=black,% you should probably change this to black before printing
	urlcolor=black,%
	pdfstartview=FitH%
}

%%%%%%%%%%%%%%%%%%%%%%%%%%%%%%%%%%%%%%%%%%%%%%%%
% Codeblocks
% Props til @ramlov
%%%%%%%%%%%%%%%%%%%%%%%%%%%%%%%%%%%%%%%%%%%%%%%%
\usepackage{listings}
\usepackage{xcolor}

%New colors defined below
\definecolor{codegreen}{rgb}{0,0.8,0}
\definecolor{codegray}{rgb}{0.65,0.65,0.65}
\definecolor{codepurple}{rgb}{0.58,0,0.82}
\definecolor{codeblue}{rgb}{0.5, 0.5, 0.9}
\definecolor{backcolour}{rgb}{0.90, 0.90, 0.9}

%Code listing style named "mystyle"
\lstdefinestyle{mystyle}{
  backgroundcolor=\color{backcolour},   
  commentstyle=\color{codegreen},
  keywordstyle=\color{magenta},
  numberstyle=\color{codegray},
  stringstyle=\color{codepurple},
  basicstyle=\ttfamily\footnotesize,
  breakatwhitespace=false,         
  breaklines=true,                 
  captionpos=b,                    
  keepspaces=true,                 
  numbers=left,                    
  numbersep=5pt,                  
  showspaces=false,                
  showstringspaces=false,
  showtabs=false,                  
  tabsize=2
}
\lstset{style=mystyle}

%%% Eksempel:
% \begin{lstlisting}[language=Python, caption=this is a caption]
    % code line 1
    % code line 2
% \end{lstlisting}