%%
%% Denne fil er til at lave macroer som kan kaldes i andre filer, hvis denne fil er defineret
%% Filen defineres med: %%
%% Denne fil er til at lave macroer som kan kaldes i andre filer, hvis denne fil er defineret
%% Filen defineres med: %%
%% Denne fil er til at lave macroer som kan kaldes i andre filer, hvis denne fil er defineret
%% Filen defineres med: %%
%% Denne fil er til at lave macroer som kan kaldes i andre filer, hvis denne fil er defineret
%% Filen defineres med: \input{setup/macros.tex}
%%

%%%%%%%%%%%%%%%%%%%%%%%%%%%%%%%%%%%%%%%%%%%%%%%%
% Reference-macroer
%%%%%%%%%%%%%%%%%%%%%%%%%%%%%%%%%%%%%%%%%%%%%%%%

% myRef - Reference med angivelse af figur/tabel
\newcommand{\myRef}[2]{\textit{{#1} \ref{#2}}}

% myRefPage - Reference med angivelse af figur/tabel og sidetal
\newcommand{\myRefPage}[2]{\textit{{#1} \ref{#2} on page \pageref{#2}}}

%%%%%%%%%%%%%%%%%%%%%%%%%%%%%%%%%%%%%%%%%%%%%%%%
% Misc-macroer
%%%%%%%%%%%%%%%%%%%%%%%%%%%%%%%%%%%%%%%%%%%%%%%%

% myTodayKort - Print af dags dato på kort format (DD/MMM-YYYY)
\newcommand{\myTodayKort}{\ifnum\number\day<10 0\fi \number\day /\ifcase \month \or jan\or feb\or mar\or apr\or may \or jun\or jul\or aug\or sep\or oct\or nov\or dec\fi -\number \year} 

% myTodayLang - Print af dags dato på langt format (DD. MMMM YYYY)(engelske måneder)
\newcommand{\myTodayLang}{\ifnum\number\day<10 0\fi \number\day .\space\ifcase \month \or January\or February\or March\or April\or May \or June\or July\or August\or September\or October\or November\or December\fi \space\number \year} 

% myAppendixPageNumber - Nulstiller sidetal, så det specificeres for det enkelte bilag og formaterer sidetallet med prefix som sættes med parameteren.
\newcommand{\myAppendixPageNumbering}[1]{%
    \pagenumbering{arabic} % resets page-counter to 1
    \renewcommand*{\thepage}{{#1}\arabic{page}} 
}


\clearpage


%%%%%%%%%%%%%%%%%%%%%%%%%%%%%%%%%%%%%%%%%%%%%%%%
% Macros for the titlepage
%%%%%%%%%%%%%%%%%%%%%%%%%%%%%%%%%%%%%%%%%%%%%%%%
%Creates the aau titlepage
\newcommand{\aautitlepage}[3]{%
  {
    %set up various length
    \ifx\titlepageleftcolumnwidth\undefined
      \newlength{\titlepageleftcolumnwidth}
      \newlength{\titlepagerightcolumnwidth}
    \fi
    \setlength{\titlepageleftcolumnwidth}{0.48\textwidth-\tabcolsep}
    \setlength{\titlepagerightcolumnwidth}{\textwidth-2\tabcolsep-\titlepageleftcolumnwidth}
    %create title page
    \thispagestyle{empty}
    \noindent%
    \begin{tabular}{@{}ll@{}}
      \parbox{\titlepageleftcolumnwidth}{
        \iflanguage{danish}{%
          \includegraphics[width=\titlepageleftcolumnwidth-5mm]{media/AAUgraphics/aau_logo_da}
        }{%
          \includegraphics[width=\titlepageleftcolumnwidth-5mm]{media/AAUgraphics/aau_logo_en}
        }
      } &
      \parbox{\titlepagerightcolumnwidth}{\raggedleft\sf\small
        #2
      }\bigskip\\
       #1 &
      \parbox[t]{\titlepagerightcolumnwidth}{%
      \textbf{Abstract:}\bigskip\par
        \fbox{\parbox{\titlepagerightcolumnwidth-2\fboxsep-2\fboxrule}{%
          #3
        }}
      }\\
    \end{tabular}
    \vfill \bigskip
    \iflanguage{danish}{\noindent{\footnotesize\emph{Rapportens indhold er frit tilgængeligt, men offentliggørelse\\ (med kildeangivelse) må kun ske efter aftale med forfatterne.}}
    }{\noindent{\footnotesize\emph{The content of this report is freely available, but publication\\ (with reference) may only be pursued due to agreement with the author.}}
    }
    \clearpage
  }
}

%Create english project info
\newcommand{\englishprojectinfo}[8]{%
  \parbox[t]{\titlepageleftcolumnwidth}{
    \textbf{Title:}\\ #1\bigskip\par
    \textbf{Theme:}\\ #2\bigskip\par
    \textbf{Project Period:}\\ #3\bigskip\par
    \textbf{Project Group:}\\ #4\bigskip\par
    \textbf{Participant(s):}\\ #5\bigskip\par
    \textbf{Supervisor(s):}\\ #6\bigskip\par
    \textbf{Copies:} #7\bigskip\par
    \textbf{Number of pages:} \pageref{myLastPage}\bigskip\par
    \textbf{Date of Completion:}\\ #8
  }
}

%Create danish project info
\newcommand{\danishprojectinfo}[8]{%
  \parbox[t]{\titlepageleftcolumnwidth}{
    \textbf{Titel:}\\ #1\bigskip\par
    \textbf{Tema:}\\ #2\bigskip\par
    \textbf{Projektperiode:}\\ #3\bigskip\par
    \textbf{Projektgruppe:}\\ #4\bigskip\par
    \textbf{Deltager(e):}\\ #5\bigskip\par
    \textbf{Vejleder(e):}\\ #6\bigskip\par
    \textbf{Oplagstal:} #7\bigskip\par
    \textbf{Sidetal:} \pageref{myLastPage}\bigskip\par
    \textbf{Afleveringsdato:}\\ #8
  }
}

%%%%%%%%%%%%%%%%%%%%%%%%%%%%%%%%%%%%%%%%%%%%%%%%
% An example environment
%%%%%%%%%%%%%%%%%%%%%%%%%%%%%%%%%%%%%%%%%%%%%%%%
\theoremheaderfont{\normalfont\bfseries}
\theorembodyfont{\normalfont}
\theoremstyle{break}
\def\theoremframecommand{{\color{gray!50}\vrule width 5pt \hspace{5pt}}}
\newshadedtheorem{exa}{Example}[chapter]
\newenvironment{example}[1]{%
		\begin{exa}[#1]
}{%
		\end{exa}
}
%%

%%%%%%%%%%%%%%%%%%%%%%%%%%%%%%%%%%%%%%%%%%%%%%%%
% Reference-macroer
%%%%%%%%%%%%%%%%%%%%%%%%%%%%%%%%%%%%%%%%%%%%%%%%

% myRef - Reference med angivelse af figur/tabel
\newcommand{\myRef}[2]{\textit{{#1} \ref{#2}}}

% myRefPage - Reference med angivelse af figur/tabel og sidetal
\newcommand{\myRefPage}[2]{\textit{{#1} \ref{#2} on page \pageref{#2}}}

%%%%%%%%%%%%%%%%%%%%%%%%%%%%%%%%%%%%%%%%%%%%%%%%
% Misc-macroer
%%%%%%%%%%%%%%%%%%%%%%%%%%%%%%%%%%%%%%%%%%%%%%%%

% myTodayKort - Print af dags dato på kort format (DD/MMM-YYYY)
\newcommand{\myTodayKort}{\ifnum\number\day<10 0\fi \number\day /\ifcase \month \or jan\or feb\or mar\or apr\or may \or jun\or jul\or aug\or sep\or oct\or nov\or dec\fi -\number \year} 

% myTodayLang - Print af dags dato på langt format (DD. MMMM YYYY)(engelske måneder)
\newcommand{\myTodayLang}{\ifnum\number\day<10 0\fi \number\day .\space\ifcase \month \or January\or February\or March\or April\or May \or June\or July\or August\or September\or October\or November\or December\fi \space\number \year} 

% myAppendixPageNumber - Nulstiller sidetal, så det specificeres for det enkelte bilag og formaterer sidetallet med prefix som sættes med parameteren.
\newcommand{\myAppendixPageNumbering}[1]{%
    \pagenumbering{arabic} % resets page-counter to 1
    \renewcommand*{\thepage}{{#1}\arabic{page}} 
}


\clearpage


%%%%%%%%%%%%%%%%%%%%%%%%%%%%%%%%%%%%%%%%%%%%%%%%
% Macros for the titlepage
%%%%%%%%%%%%%%%%%%%%%%%%%%%%%%%%%%%%%%%%%%%%%%%%
%Creates the aau titlepage
\newcommand{\aautitlepage}[3]{%
  {
    %set up various length
    \ifx\titlepageleftcolumnwidth\undefined
      \newlength{\titlepageleftcolumnwidth}
      \newlength{\titlepagerightcolumnwidth}
    \fi
    \setlength{\titlepageleftcolumnwidth}{0.48\textwidth-\tabcolsep}
    \setlength{\titlepagerightcolumnwidth}{\textwidth-2\tabcolsep-\titlepageleftcolumnwidth}
    %create title page
    \thispagestyle{empty}
    \noindent%
    \begin{tabular}{@{}ll@{}}
      \parbox{\titlepageleftcolumnwidth}{
        \iflanguage{danish}{%
          \includegraphics[width=\titlepageleftcolumnwidth-5mm]{media/AAUgraphics/aau_logo_da}
        }{%
          \includegraphics[width=\titlepageleftcolumnwidth-5mm]{media/AAUgraphics/aau_logo_en}
        }
      } &
      \parbox{\titlepagerightcolumnwidth}{\raggedleft\sf\small
        #2
      }\bigskip\\
       #1 &
      \parbox[t]{\titlepagerightcolumnwidth}{%
      \textbf{Abstract:}\bigskip\par
        \fbox{\parbox{\titlepagerightcolumnwidth-2\fboxsep-2\fboxrule}{%
          #3
        }}
      }\\
    \end{tabular}
    \vfill \bigskip
    \iflanguage{danish}{\noindent{\footnotesize\emph{Rapportens indhold er frit tilgængeligt, men offentliggørelse\\ (med kildeangivelse) må kun ske efter aftale med forfatterne.}}
    }{\noindent{\footnotesize\emph{The content of this report is freely available, but publication\\ (with reference) may only be pursued due to agreement with the author.}}
    }
    \clearpage
  }
}

%Create english project info
\newcommand{\englishprojectinfo}[8]{%
  \parbox[t]{\titlepageleftcolumnwidth}{
    \textbf{Title:}\\ #1\bigskip\par
    \textbf{Theme:}\\ #2\bigskip\par
    \textbf{Project Period:}\\ #3\bigskip\par
    \textbf{Project Group:}\\ #4\bigskip\par
    \textbf{Participant(s):}\\ #5\bigskip\par
    \textbf{Supervisor(s):}\\ #6\bigskip\par
    \textbf{Copies:} #7\bigskip\par
    \textbf{Number of pages:} \pageref{myLastPage}\bigskip\par
    \textbf{Date of Completion:}\\ #8
  }
}

%Create danish project info
\newcommand{\danishprojectinfo}[8]{%
  \parbox[t]{\titlepageleftcolumnwidth}{
    \textbf{Titel:}\\ #1\bigskip\par
    \textbf{Tema:}\\ #2\bigskip\par
    \textbf{Projektperiode:}\\ #3\bigskip\par
    \textbf{Projektgruppe:}\\ #4\bigskip\par
    \textbf{Deltager(e):}\\ #5\bigskip\par
    \textbf{Vejleder(e):}\\ #6\bigskip\par
    \textbf{Oplagstal:} #7\bigskip\par
    \textbf{Sidetal:} \pageref{myLastPage}\bigskip\par
    \textbf{Afleveringsdato:}\\ #8
  }
}

%%%%%%%%%%%%%%%%%%%%%%%%%%%%%%%%%%%%%%%%%%%%%%%%
% An example environment
%%%%%%%%%%%%%%%%%%%%%%%%%%%%%%%%%%%%%%%%%%%%%%%%
\theoremheaderfont{\normalfont\bfseries}
\theorembodyfont{\normalfont}
\theoremstyle{break}
\def\theoremframecommand{{\color{gray!50}\vrule width 5pt \hspace{5pt}}}
\newshadedtheorem{exa}{Example}[chapter]
\newenvironment{example}[1]{%
		\begin{exa}[#1]
}{%
		\end{exa}
}
%%

%%%%%%%%%%%%%%%%%%%%%%%%%%%%%%%%%%%%%%%%%%%%%%%%
% Reference-macroer
%%%%%%%%%%%%%%%%%%%%%%%%%%%%%%%%%%%%%%%%%%%%%%%%

% myRef - Reference med angivelse af figur/tabel
\newcommand{\myRef}[2]{\textit{{#1} \ref{#2}}}

% myRefPage - Reference med angivelse af figur/tabel og sidetal
\newcommand{\myRefPage}[2]{\textit{{#1} \ref{#2} on page \pageref{#2}}}

%%%%%%%%%%%%%%%%%%%%%%%%%%%%%%%%%%%%%%%%%%%%%%%%
% Misc-macroer
%%%%%%%%%%%%%%%%%%%%%%%%%%%%%%%%%%%%%%%%%%%%%%%%

% myTodayKort - Print af dags dato på kort format (DD/MMM-YYYY)
\newcommand{\myTodayKort}{\ifnum\number\day<10 0\fi \number\day /\ifcase \month \or jan\or feb\or mar\or apr\or may \or jun\or jul\or aug\or sep\or oct\or nov\or dec\fi -\number \year} 

% myTodayLang - Print af dags dato på langt format (DD. MMMM YYYY)(engelske måneder)
\newcommand{\myTodayLang}{\ifnum\number\day<10 0\fi \number\day .\space\ifcase \month \or January\or February\or March\or April\or May \or June\or July\or August\or September\or October\or November\or December\fi \space\number \year} 

% myAppendixPageNumber - Nulstiller sidetal, så det specificeres for det enkelte bilag og formaterer sidetallet med prefix som sættes med parameteren.
\newcommand{\myAppendixPageNumbering}[1]{%
    \pagenumbering{arabic} % resets page-counter to 1
    \renewcommand*{\thepage}{{#1}\arabic{page}} 
}


\clearpage


%%%%%%%%%%%%%%%%%%%%%%%%%%%%%%%%%%%%%%%%%%%%%%%%
% Macros for the titlepage
%%%%%%%%%%%%%%%%%%%%%%%%%%%%%%%%%%%%%%%%%%%%%%%%
%Creates the aau titlepage
\newcommand{\aautitlepage}[3]{%
  {
    %set up various length
    \ifx\titlepageleftcolumnwidth\undefined
      \newlength{\titlepageleftcolumnwidth}
      \newlength{\titlepagerightcolumnwidth}
    \fi
    \setlength{\titlepageleftcolumnwidth}{0.48\textwidth-\tabcolsep}
    \setlength{\titlepagerightcolumnwidth}{\textwidth-2\tabcolsep-\titlepageleftcolumnwidth}
    %create title page
    \thispagestyle{empty}
    \noindent%
    \begin{tabular}{@{}ll@{}}
      \parbox{\titlepageleftcolumnwidth}{
        \iflanguage{danish}{%
          \includegraphics[width=\titlepageleftcolumnwidth-5mm]{media/AAUgraphics/aau_logo_da}
        }{%
          \includegraphics[width=\titlepageleftcolumnwidth-5mm]{media/AAUgraphics/aau_logo_en}
        }
      } &
      \parbox{\titlepagerightcolumnwidth}{\raggedleft\sf\small
        #2
      }\bigskip\\
       #1 &
      \parbox[t]{\titlepagerightcolumnwidth}{%
      \textbf{Abstract:}\bigskip\par
        \fbox{\parbox{\titlepagerightcolumnwidth-2\fboxsep-2\fboxrule}{%
          #3
        }}
      }\\
    \end{tabular}
    \vfill \bigskip
    \iflanguage{danish}{\noindent{\footnotesize\emph{Rapportens indhold er frit tilgængeligt, men offentliggørelse\\ (med kildeangivelse) må kun ske efter aftale med forfatterne.}}
    }{\noindent{\footnotesize\emph{The content of this report is freely available, but publication\\ (with reference) may only be pursued due to agreement with the author.}}
    }
    \clearpage
  }
}

%Create english project info
\newcommand{\englishprojectinfo}[8]{%
  \parbox[t]{\titlepageleftcolumnwidth}{
    \textbf{Title:}\\ #1\bigskip\par
    \textbf{Theme:}\\ #2\bigskip\par
    \textbf{Project Period:}\\ #3\bigskip\par
    \textbf{Project Group:}\\ #4\bigskip\par
    \textbf{Participant(s):}\\ #5\bigskip\par
    \textbf{Supervisor(s):}\\ #6\bigskip\par
    \textbf{Copies:} #7\bigskip\par
    \textbf{Number of pages:} \pageref{myLastPage}\bigskip\par
    \textbf{Date of Completion:}\\ #8
  }
}

%Create danish project info
\newcommand{\danishprojectinfo}[8]{%
  \parbox[t]{\titlepageleftcolumnwidth}{
    \textbf{Titel:}\\ #1\bigskip\par
    \textbf{Tema:}\\ #2\bigskip\par
    \textbf{Projektperiode:}\\ #3\bigskip\par
    \textbf{Projektgruppe:}\\ #4\bigskip\par
    \textbf{Deltager(e):}\\ #5\bigskip\par
    \textbf{Vejleder(e):}\\ #6\bigskip\par
    \textbf{Oplagstal:} #7\bigskip\par
    \textbf{Sidetal:} \pageref{myLastPage}\bigskip\par
    \textbf{Afleveringsdato:}\\ #8
  }
}

%%%%%%%%%%%%%%%%%%%%%%%%%%%%%%%%%%%%%%%%%%%%%%%%
% An example environment
%%%%%%%%%%%%%%%%%%%%%%%%%%%%%%%%%%%%%%%%%%%%%%%%
\theoremheaderfont{\normalfont\bfseries}
\theorembodyfont{\normalfont}
\theoremstyle{break}
\def\theoremframecommand{{\color{gray!50}\vrule width 5pt \hspace{5pt}}}
\newshadedtheorem{exa}{Example}[chapter]
\newenvironment{example}[1]{%
		\begin{exa}[#1]
}{%
		\end{exa}
}
%%

%%%%%%%%%%%%%%%%%%%%%%%%%%%%%%%%%%%%%%%%%%%%%%%%
% Reference-macroer
%%%%%%%%%%%%%%%%%%%%%%%%%%%%%%%%%%%%%%%%%%%%%%%%

% myRef - Reference med angivelse af figur/tabel
\newcommand{\myRef}[2]{\textit{{#1} \ref{#2}}}

% myRefPage - Reference med angivelse af figur/tabel og sidetal
\newcommand{\myRefPage}[2]{\textit{{#1} \ref{#2} on page \pageref{#2}}}

%%%%%%%%%%%%%%%%%%%%%%%%%%%%%%%%%%%%%%%%%%%%%%%%
% Misc-macroer
%%%%%%%%%%%%%%%%%%%%%%%%%%%%%%%%%%%%%%%%%%%%%%%%

% myTodayKort - Print af dags dato på kort format (DD/MMM-YYYY)
\newcommand{\myTodayKort}{\ifnum\number\day<10 0\fi \number\day /\ifcase \month \or jan\or feb\or mar\or apr\or may \or jun\or jul\or aug\or sep\or oct\or nov\or dec\fi -\number \year} 

% myTodayLang - Print af dags dato på langt format (DD. MMMM YYYY)(engelske måneder)
\newcommand{\myTodayLang}{\ifnum\number\day<10 0\fi \number\day .\space\ifcase \month \or January\or February\or March\or April\or May \or June\or July\or August\or September\or October\or November\or December\fi \space\number \year} 

% myAppendixPageNumber - Nulstiller sidetal, så det specificeres for det enkelte bilag og formaterer sidetallet med prefix som sættes med parameteren.
\newcommand{\myAppendixPageNumbering}[1]{%
    \pagenumbering{arabic} % resets page-counter to 1
    \renewcommand*{\thepage}{{#1}\arabic{page}} 
}


\clearpage


%%%%%%%%%%%%%%%%%%%%%%%%%%%%%%%%%%%%%%%%%%%%%%%%
% Macros for the titlepage
%%%%%%%%%%%%%%%%%%%%%%%%%%%%%%%%%%%%%%%%%%%%%%%%
%Creates the aau titlepage
\newcommand{\aautitlepage}[3]{%
  {
    %set up various length
    \ifx\titlepageleftcolumnwidth\undefined
      \newlength{\titlepageleftcolumnwidth}
      \newlength{\titlepagerightcolumnwidth}
    \fi
    \setlength{\titlepageleftcolumnwidth}{0.48\textwidth-\tabcolsep}
    \setlength{\titlepagerightcolumnwidth}{\textwidth-2\tabcolsep-\titlepageleftcolumnwidth}
    %create title page
    \thispagestyle{empty}
    \noindent%
    \begin{tabular}{@{}ll@{}}
      \parbox{\titlepageleftcolumnwidth}{
        \iflanguage{danish}{%
          \includegraphics[width=\titlepageleftcolumnwidth-5mm]{media/AAUgraphics/aau_logo_da}
        }{%
          \includegraphics[width=\titlepageleftcolumnwidth-5mm]{media/AAUgraphics/aau_logo_en}
        }
      } &
      \parbox{\titlepagerightcolumnwidth}{\raggedleft\sf\small
        #2
      }\bigskip\\
       #1 &
      \parbox[t]{\titlepagerightcolumnwidth}{%
      \textbf{Abstract:}\bigskip\par
        \fbox{\parbox{\titlepagerightcolumnwidth-2\fboxsep-2\fboxrule}{%
          #3
        }}
      }\\
    \end{tabular}
    \vfill \bigskip
    \iflanguage{danish}{\noindent{\footnotesize\emph{Rapportens indhold er frit tilgængeligt, men offentliggørelse\\ (med kildeangivelse) må kun ske efter aftale med forfatterne.}}
    }{\noindent{\footnotesize\emph{The content of this report is freely available, but publication\\ (with reference) may only be pursued due to agreement with the author.}}
    }
    \clearpage
  }
}

%Create english project info
\newcommand{\englishprojectinfo}[8]{%
  \parbox[t]{\titlepageleftcolumnwidth}{
    \textbf{Title:}\\ #1\bigskip\par
    \textbf{Theme:}\\ #2\bigskip\par
    \textbf{Project Period:}\\ #3\bigskip\par
    \textbf{Project Group:}\\ #4\bigskip\par
    \textbf{Participant(s):}\\ #5\bigskip\par
    \textbf{Supervisor(s):}\\ #6\bigskip\par
    \textbf{Copies:} #7\bigskip\par
    \textbf{Number of pages:} \pageref{myLastPage}\bigskip\par
    \textbf{Date of Completion:}\\ #8
  }
}

%Create danish project info
\newcommand{\danishprojectinfo}[8]{%
  \parbox[t]{\titlepageleftcolumnwidth}{
    \textbf{Titel:}\\ #1\bigskip\par
    \textbf{Tema:}\\ #2\bigskip\par
    \textbf{Projektperiode:}\\ #3\bigskip\par
    \textbf{Projektgruppe:}\\ #4\bigskip\par
    \textbf{Deltager(e):}\\ #5\bigskip\par
    \textbf{Vejleder(e):}\\ #6\bigskip\par
    \textbf{Oplagstal:} #7\bigskip\par
    \textbf{Sidetal:} \pageref{myLastPage}\bigskip\par
    \textbf{Afleveringsdato:}\\ #8
  }
}

%%%%%%%%%%%%%%%%%%%%%%%%%%%%%%%%%%%%%%%%%%%%%%%%
% An example environment
%%%%%%%%%%%%%%%%%%%%%%%%%%%%%%%%%%%%%%%%%%%%%%%%
\theoremheaderfont{\normalfont\bfseries}
\theorembodyfont{\normalfont}
\theoremstyle{break}
\def\theoremframecommand{{\color{gray!50}\vrule width 5pt \hspace{5pt}}}
\newshadedtheorem{exa}{Example}[chapter]
\newenvironment{example}[1]{%
		\begin{exa}[#1]
}{%
		\end{exa}
}