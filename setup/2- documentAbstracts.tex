%%% Adjust the width of the abstract box
% Note: the AAU logo scales with this as well
% Suggested width range between 0.550 and 0.625
\def\titlepageAbstractWidth{0.575} % <------- Adjust this to allow for longer/shorter Abstract.

\def\myAbstractEN{%
% Engelsk Abstract:
This is where the abstract for the project is written.\\

Et abstract/resumé er en kort sammenfatning af hele rapportens indhold med fokus på problemstilling, anvendte metoder, resultater og konklusion.
Det er ud fra resuméet at læseren vurderer, om rapporten har interesse eller ej, og det skal kunne læses
uafhængigt af selve rapporten.
Et resumé fylder sjældent mere end ½-1 A4 side.
Der skal som hovedregel altid skrives både et dansk resumé samt et engelsk (abstract). \\
Kilde: \url{https://www.sdu.dk/-/media/files/om_sdu/institutter/ikbm/diplom+biotek/guide+til+rapportskrivning.pdf}
}

\def\myAbstractDA{%
% Dansk resumé:
Her skrives resuméet til projektet.\\

Et abstract/resumé er en kort sammenfatning af hele rapportens indhold med fokus på problemstilling, anvendte metoder, resultater og konklusion.
Det er ud fra resuméet at læseren vurderer, om rapporten har interesse eller ej, og det skal kunne læses
uafhængigt af selve rapporten.
Et resumé fylder sjældent mere end ½-1 A4 side.
Der skal som hovedregel altid skrives både et dansk resumé samt et engelsk (abstract).\\
Kilde: \url{https://www.sdu.dk/-/media/files/om_sdu/institutter/ikbm/diplom+biotek/guide+til+rapportskrivning.pdf}
}